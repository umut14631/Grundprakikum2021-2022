\section{Diskussion}

\begin{flushleft}
    Bei der statischen Methode fällt auf, dass alle einen ähnlichen Verlauf darstellen z.B. die Graphen steigen exponentiell auf und flachen sich mit der Zeit ab. 
    Bei der dynamischen Methode spiegelt sich der periodische Verlauf vom Erhitzen und Abkühlen wieder. 
    Beim Erhitzen sind die Peaks der Amplituden deutlich stärker, beim Abkühlen schwächer, jedoch sieht die Art und Weise des Verlaufs ähnlich aus.
    Folgende Abweichung von den Literaturwerten \cite{a7} \cite{a2} \cite{a6} \cite{a4} \cite{a3} \cite{a5} liegen vor:
\end{flushleft}

\begin{table}
    \centering
    \caption{}
    \label{Tabelle}
    \begin{tabular} {c  c  c  c}
        \toprule
        {$ $} &
        {$ Abweichungen $} \\
        \midrule
        Messingstab   & 20,18\,\% \\
        Aluminiumstab & 12,38\,\% \\
        Edelstahlstab & 34,60\,\% \\
        \bottomrule
    \end{tabular}  
\end{table}

\begin{flushleft}
    Zum einen lässt sich die Fehlerquelle auf das Ablesen hinweisen, da die Werte der Temperaturen sowie die Messung de Amplituden aus den Grafiken entnommen werden.
    Zum anderen stellt das Messgerät ebenso eine Fehlerquelle dar, da sie veraltet ist und dadurch die Messswerte abweichung können. 
    Das Umschalten zur richtigen Zeit, sowie das Umlegen der Schalter, führt zur Zeitverlusten.
    Die Proben wurden ebenso nicht komplett von der Isolation ummantelt, was dazu führt, dass die Stäbe trotz Isolation Energie verlieren.
    Da es sich um ein nicht abgeschlossenes System handelt, nähert sich die Temperatur der Raumtemperatur an.
    Eine weitere Möglichkeit wäre, dass das Messgerät nach einer Zeit selber Wärme abgibt, da es sich um ein recht altes Gerät handelt, wodurch dann Fehler bei der Messung entstehen können.
    Hierbei handelt es sich mehr um systematische Fehler, aber dennoch sind die Werte akzeptabel.
\end{flushleft}
\section{Theorie}

\begin{flushleft}
    Ein Temperaturangleich eines Körpers welcher nicht im Temperaturgleichgewicht ist, kann über drei verschiedene Wärmetransportwege stattfinden. 
    Diese sind Konvektion, Wärmestrahlung und Wärmeleitung. 
    Die Wärmeleitung ist der Wärmetransport, auf welchen man sich in diesem Versuch konzentiert.
    Dieser Wärmetransport erfolgt über Phononen und frei bewegliche Elektronen, dies bezieht sich speziell auf Festkörper.
    Es wird ein Stab betrachtet, mt der Länge $ L $, der Querschnittsfläche $ A $, der Materialdichte $ \rho $ und einer spezifischen Wärme $ c $.
    Wenn dieser Stab an einem Ende erhitzt wird, oder wärmer ist als der Rest, so fließt eine Wärmemenge 
\end{flushleft}

\begin{equation}
    dQ = -\kappa \cdot A \cdot \frac{\partial T}{\partial x}dt \label{1}
\end{equation}

\begin{flushleft}
    in einer bestimmten Zeit $ dt $ durch die Querschnittfläche $ A $.
    Das Minus in der Formel (\ref{1}) steht für die Konvektion, also dem Wärmestrom in die kältere Richtung und das $ \kappa $ für die materialabhängige Wärmeleitfähigkeit des Körpers.
    Die Wärmestromdichte $ j_{w} $ wird berechnet durch die Formel
\end{flushleft}

\begin{equation}
    j_{w} = -\kappa \cdot \frac{\partial T}{\partial x}.\label{2}
\end{equation}
 
\begin{flushleft}
    Durch die Kontinuitätsgleichung kann man die eindimensionale Wärmeleitungsgleichung 
\end{flushleft}

\begin{equation}
    \frac{\partial T}{\partial t} = \frac{\kappa}{\rho \cdot c} \cdot \frac{\partial^2 T} {\partial x^2}\label{3}
\end{equation}

\begin{flushleft}
    aufstellen, welche die Temperaturverteilung in seiner räumlichen und zeitlichen Entwicklung beschreibt.
    Die Temperaturleitfähigkeit wird durch die Größe $ \frac{\kappa}{\rho \cdot c} = \sigma_{\text{T}}  $ beschrieben und gibt an wie schnell der Temperaturunterschied ausgeglichen wird.
    Entscheident für die Lösung der der Wärmeleitungsgleichung sind die Stabgeometrie und die Anfangsbedingungen. 
\end{flushleft}

\newpage

\begin{flushleft}
    Wenn ein langer Stab abwechselnd erwärmt und abgekühlt wird mit einer Periode $ T $,  
    wird aufgrund der periodischen Temperaturwechsel eine zeitliche sowie räumliche Temperaturwelle 
\end{flushleft}

\begin{equation}
    T(x,t) = T_{\text{max}} \cdot e^{-\sqrt{\frac{\omega \rho c}{2\kappa}} \cdot x} \cdot \cos\left({\omega t - \sqrt{\frac{\omega \rho c}{2\kappa}} \cdot x }\right)\label{4}
\end{equation}

\begin{flushleft}
    erzeugt und weitergegeben.
\end{flushleft}

\begin{flushleft}
    Die Temperaturwelle bewegt sich mit der Phasengeschwindigkeit 
\end{flushleft}

\begin{equation}
   \nu = \frac{\omega}{\kappa} = \omega \mathbin{/} \sqrt{\frac{\omega \rho c}{2\kappa}} = \sqrt{\frac{2\kappa \cdot \omega}{\rho \cdot c}}.
\end{equation}

\begin{flushleft}
    Das Amplitudenverhältnis $A_{\text{nah}}$ und $ A_{\text{fern}} $ an den zwei Messstellen $ x_{\text{nah}} $ und $ x_{\text{fern}} $ der Welle liefert die Dämpfung der Welle. 
    Wenn angenommen wird, dass $\omega = 2\pi \mathbin{/} T^*$ und $ \phi = 2\pi \increment t \mathbin{/} T^*  $ mit der Phase $ \phi $ und der Periodendauer $T^*$, folgt die Wärmeleitfähigkeit als
\end{flushleft}

\begin{equation}
    \kappa = \frac{\rho c (\increment x)^2 }{2 \increment t \left(A_{\text{nah}} \mathbin{/} A_{\text{left}}\right)}. \label{6}
\end{equation}
 
\begin{flushleft}
    Der Abstand der beiden Messstellen ist gegeben durch $ \increment x $ und die Phasendifferenz der Temperaturwelle zwischen den beiden Messstellen als $ \increment t $.
\end{flushleft}


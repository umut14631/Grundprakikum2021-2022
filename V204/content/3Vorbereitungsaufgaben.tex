\section{Vorbereitungsaufgaben}

\begin{flushleft}
    Die Vorbereitungsaufgabe für diesen Versuch besteht aus der Recherche von Stoffeigenschaften der drei Metalle und Wasser, die in diesem Versuch benötigt werden. 
    Die drei Metalle sind Edelstahl (V2A), Aluminium und Messing. Die Stoffeigenschaften, welche recherchiert werden sollen, sind Dichte $ \rho $, spezifische Wärme $ c $ und Wärmeleitfähigkeit $ \kappa $. 
\end{flushleft}

\begin{table}
    \centering
    \caption{}
    \label{Tabelle}
    \begin{tabular} {c||  c  c  c}
        \toprule
        {$ $} &
        {$ \rho \mathbin{/} \frac{\unit{\kilo\gram}}{\unit{\meter^3}} $} &
        {$ c \mathbin{/} \frac{\unit{\joule}}{(\unit{\kilo\gram} \cdot \unit{\kelvin})} $} &
        {$ \kappa  \mathbin{/} \frac{\unit{\watt}}{(\unit{\meter} \cdot \unit{\kelvin})} $} \\
        \midrule
        Wasser\,\,\cite{a2}  & 1000 & 4187 & 0,6  \\
        Edelstahl\,\,\cite{a3}\cite{a5}  & 8000 & 477 & 15  \\
        Aluminium\,\,\cite{a3}\cite{a4}\cite{a7}  & 2700 & 888 & 237  \\
        Messing\,\,\cite{a3}\cite{a4}\cite{a6}  & 8400 & 377 & 120  \\
    \end{tabular} 
\end{table}
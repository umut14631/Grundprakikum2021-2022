\section{Diskussion}

\begin{flushleft}
    Betrachtet man die Auswertung, weichen drei Werte stark von dem theoretischen Wert ab. 
    Wie beim ersten Versuch dargestellt stellt sich ein Wert heraus, welcher eine große Abweichung vorweist.
    Das $\mu_{1}$ wurde mithilfe von der Einhüllenden $\text{A}= \text{A}_{0} \cdot e^{-2\pi\,\mu\,t}$ berechnet.
    Zwei mögliche Gründe dafür sind das falsches Ablesen der Werte, oder die falsche Berechnung mithilfe eines Darstellungsprogramms. 
    Jedoch scheinen die abgelesenen Werte realistisch, bezogen auf die Darstellung in dem Diagramm \ref{Abbildung10}.
    Die Dadurch folgende Differenz von $ 67,196\,\unit{\ohm} $ lassen sich durch das nicht betrachten der Innenwiderstände des Oszilloskops, welche ungefähr bei $50\, \unit{\ohm}$ \cite{rohde} liegen, erläutern.
    Die Güte ist ebenfalls ein Faktor, welcher Abweichungen des theoretischen Werts vorweist. Die Breite der Resonanzkurve befindet sich im Toleranzbereich. 
    Bei genauer Betrachtung weichen viele Werte ab, dies lässt sich auf überwiegend systematische Fehler zurückweisen.
\end{flushleft}
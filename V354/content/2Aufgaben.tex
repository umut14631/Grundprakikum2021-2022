\section{Aufgaben}
\label{sec: Aufgaben}

Für diesen Versuch gibt es fünf verschiedene Arbeitsaufträge. Man beginnt mit der Untersuchung der Zeitabhängigkeit von der Amplitude einer gedämpften Schwingung und bestimmt den effektiven Dämpfungswiderstand durch die eingehenden Messungen.
Danach bestimmt man den Dämpfungswiderstand des aperiodischen Grenzfalles. Als nächstes muss die Frequenzabhängige Kondensatorspannung über einen Serienresonanzkreis gemessen werden.
Als letztes wird die Frequenzabhängigkeit von der Phase bestimmt, welche zwischen Erreger- und Kondensatorspannung vorhanden ist. 






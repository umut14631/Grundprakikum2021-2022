\section{Vorbereitungsaufgabe}

\begin{align*}
    \intertext{Die erste Vorbereitungsaufgabe besteht aus der Recherche der Literaturwerte für die Energien der $\text{Cu-K}_{\alpha}$ - und $\text{Cu-K}_{\beta}$ Linien und bei welchen Winkeln $\theta$ diese bei einem KBr-Kristall liegen.
    Der Kristall hat eine Dicke von $d = 204.1\,\text{pm}$. }
\end{align*}

\begin{flushleft}
    In der zweiten Vorbereitungsaufgabe soll eine Tabelle erstellt werden für fünf verschiedene Stoffe.
    Aufgenommen werden die Ordnungszahl Z, der Literaturwert \cite{a2} \cite{a3} der K-Kante $\text{E}^{\text{lit}}_{\text{K}}$, der Braggwinkel zu $\text{E}^{\text{lit}}_{\text{K}}$ $\theta^{\text{lit}}_{\text{K}}$ und die Abschirmkonstante $\sigma_{\text{K}}$.
    Um den Winkel $\theta^{\text{lit}}_{\text{K}}$ zu bestimmen muss in den Nenner der Formel (\ref{1}) der Wert für $\text{E}^{\text{lit}}_{\text{K}}$ eingesetzt werden und durch das Umstellen der Bragg Bedingung in Formel (\ref{7}) der Winkel bestimmt.
\end{flushleft}

\begin{table}[H]     
    \centering
    \caption{} 
    \label{Tabelle1}
    \begin{tabular} {c | c | c | c | c}
        \toprule
        {$  $} &
        {$ \text{Z} $} &
        {$ \text{E}^{\text{lit}}_{\text{K}} \mathbin{/} \text{keV}$} &
        {$ \theta^{\text{lit}}_{\text{K}} \mathbin{/} \unit{\degree} $} &
        {$ \sigma_{\text{K}} $} \\
        \midrule
        Zn & 30 & 9,65  & 18,28 & 3,96 \\
        Ga & 31 & 10,37 & 17,01 & 3,39 \\
        Br & 35 & 13,47 & 13,09 & 3,53 \\
        Sr & 38 & 16,10 & 10,96 & 3,59 \\
        Zr & 40 & 17,99 & 9,80  & 3,63 \\
        \bottomrule
    \end{tabular} 
\end{table}
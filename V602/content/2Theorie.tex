\section{Theorie}

\begin{align}
    \intertext{Um Röntgenstrahlen zu erzeugen werden Elektronen durch das Erhitzen einer Glühkathode, welche in einem elektischen Feld ist, durch eine evakuierte Röhre auf eine Anode beschleunigt.
    Beim Auftreffen der Elektronen auf der Anode entsteht Röntgenstrahlung. 
    Diese setzt sich aus der charakteristischen Röntgenstrahlung des Anodenmaterials sowie dem kontinuierlichen Bremsprektrum zusammen.
    Das Elektron wird in dem Coulombfeld des Atomkerns abgebremst, wobei ein Photon ausgesendet wird, welches die Enegrie besitzt, die das Elektron beim Abbremsen verliert.
    Dies ist der Vorgang beim Entstehen des Bremsprektrums.
    Damit das Bremsprektrum kontinuierlich ist, muss das Elektron seine ganze kinetische Energie, sowie seine Energie abgeben.
    Die minimale Wellenlänge bei vollständiger Abbremsung ergibt sich durch}
    \lambda_{\text{min}} = \frac{h \cdot \text{c}}{e_{0} \cdot \text{U}} \label{1}
    \intertext{und wird von kinetischer Energie zu Strahlungsenergie umgewandelt.
    Bei dem charakteristischen Spektrum wird das Anodenmaterial so ionisiert, dass eine Leerstelle in der innersten Schale entsteht, wodurch ein Elektron, unter der Vorraussetzung das ein Röntgenquant ausgesendet wird, aus einer äußeren Schale in die innere zurückfallen kann. 
    Die Energiedifferenz $h\nu = \text{E}_{\text{kin}} - \text{E}_{\text{n}} $ der beiden Energieniveaus ist gleich der Energie des Röntgenquants, wodurch das charakteristische Spektrum aus scharfen Linien besteht und die Energie charakteristisch für das Anodenmaterial der Röntgenröhre ist.
    Diese Linien werden durch die Schale (K,L,M) und $\alpha$ oder $\beta$ gekennzeichnet, beispielsweise $\text{K}_{\alpha}$, $\text{K}_{\beta}$.
    Bei einem Mehrelektronenatom verringert sich die Coulomb Anziehung aufgrund des Schirmens durch die Hüllenelektronen und die Wechselwirkung der Elektronen untereinander, wodurch sich die Bindungsenergie eines Elektrons auf der n-ten Schale durch}
    \text{E}_{\text{n}} = - \text{R}_{\infty} \text{z}^2_{\text{eff}} \cdot \frac{1}{\text{n}^2} \label{2}
    \intertext{berechnen lässt. $\text{R}_{\infty} = 13,6\,\text{eV}$ ist die Rydbergenergie, $\sigma$ die Abschirmkonstante und $\text{z}_{\text{eff}} = z - \sigma $ die effektive Kernladung, welche den Abschirmeffekt berücksichtigt.
    Durch die Formel (\ref{2})  lässt sich die Energie $\text{E}_{\text{K}\alpha}$, wobei die Abschirmkonstante für jedes Elektron unterschiedlich ist und empirisch bestimmt wird.}
    \text{E}_{\text{K}\alpha} =  \text{R}_{\infty} (\text{z} - \sigma_{1})^2 \cdot \frac{1}{1^{2}} - \text{R}_{\infty} (\text{z} - \sigma_{2})^2 \cdot \frac{1}{2^{2}} \label{3}\\ \notag
\end{align}

\begin{align}
    \intertext{Durch den Bahndrehimpuls sowie den Elektronenspin der äußeren Elektronen, besitzen nicht alle die selbe Bindungsenergie, was zur Folge hat, dass die charakteristischen Linien in einer Reihe eng beieinander liegen aufgelöst sind.
    Diese sind jedoch nicht aufspaltbar in diesem Versuch.
    Beobachtbar sind hierbei die Überlagerung der Bremsstrahlung der $\text{Cu-K}_{\alpha}$ - und $\text{Cu-K}_{\beta}$ Linien die von der Kupferanode ausgehen. 
    Der Comptoneffekt und der Photoeffekt sind zwei dominate Prozesse die bei der Absorption von Röntgenstrahlung unter $1\,\text{MeV}$ auftreten.
    Durch zunehmende Energie wird der Absorptionskoeffizient kleiner, steigt jedoch sprunghaft an wenn die Bindungsenergie eines Elektrons aus der nächst inneren Schale kleiner ist als die Energie des Photons.
    Die Bindungsenergie des Elektrons ist nahezu identisch mit den Absorptionskanten $\text{h}\nu_{\text{abs}} = \text{E}_{n} - \text{E}_{\infty}$, wobei die Energien je nach Schale als K-, L-, M- Absorptionskante bezeichnet werden.
    Bei Betrachtung der Feinstruktur fällt auf, dass die K-Kante nur einmal vorhanden ist, wobei die L-Kante drei verschiedene besitzt.
    Durch Berechnung der Bindungsenergie $\text{E}_{\text{n,j}}$ eines Elektrons mit Hilfe der Sommerfeldschen Feinstrukturformel, werden diese Feinstrukturen berücksichtigt.}
    \text{E}_{\text{n,j}} = - \text{R}_{\infty} \left(\text{z}^2_{\text{eff,1}} \cdot \frac{1}{n^2} + \alpha^2 \text{z}^4_{\text{eff,2}} \cdot \frac{1}{n^3} \left( \frac{1}{j+\frac{1}{2}} - \frac{3}{4n} \right)\right) \label{4} \\
    \intertext{Hierbei steht $\text{R}_{\infty}$ für die Rydbergenergie, $\text{z}_{\text{eff}}$ für die effektive Kernladung, $\alpha$ für die Sommerfeldsche Feinstrukturkonstante, n für die Hauptquantenzahl und j für den Gesamtdrehimpuls des Elektrons.
    Die Abschirmkonstante $\sigma_{\text{K, abs}}$ kann durch die Sommerfeldsche Feinstrukturformel für ein Elektron aus der K-Schale mit der Hauptquantenzahl eins bestimmt werden.}
    \sigma_{\text{K}} = \text{Z} - \sqrt{\frac{ \text{E}_{\text{K}}}{\text{R}_{\infty}} - \frac{\alpha^2 \text{Z}^4}{4}}\,. \label{5} \\
    \intertext{Die Feinstrukturspaltung der K-Schale wird durch den zweiten Term unter der Wurzel berücksichtigt.
    Um die Abschirmkonstante $\sigma_{\text{L}}$ zu bestimmen müssen die Abschirmzahlen jedes Elektrons berücksichtigt werden, was bedeutet das dabei ebenso die Feinstrukturen berücksichtigt werden.
    Die Berechnung der Abschirmkonstante kann jedoch durch die Energiedifferenz $\increment \text{E}_{\text{L}}$ zweier L-Kanten bestimmt werden.}
    \sigma_{L} = \text{Z} - \left(\frac{4}{\alpha} \sqrt{\frac{\increment \text{E}_{\text{L}}}{\text{R}_{\infty}}} - \frac{5\increment \text{E}_{\text{L}}}{\text{R}_{\infty}} \right)^{\frac{1}{2}} \left(1+\frac{19}{32}\alpha^2 \frac{\increment \text{E}_{\text{L}}}{\text{R}_{\infty}}\right)^{\frac{1}{2}} \label{6} \\
    \intertext{wobei die Energiedifferenz $\increment \text{E}_{\text{L}} = \text{E}_{\text{L}_{II}} - \text{E}_{\text{L}_{III}} $, Z die Ordnungszahl, $\text{R}_{\infty}$ die Rydbergenergie und $\alpha$ die Feinstrukturkonstante ist.} \notag
\end{align}

\begin{align}
    \intertext{Die Braggsche Reflexion ist ein experimenteller Weg die Energie E sowie die Wellenlänge $\lambda$ der Röntgenstrahlung zu bestimmen.
    Dabei wird das Röntgenlicht auf eine, wie in diesem Versuch vorhandenen, LiF-Kristall geworfen, wobei der Kristall wie ein dreidimensionales Gitter funktioniert und die Photonen an jedem Atom des Kristalls gebeugt werden.
    Durch das interferieren der Röntgenstrahlen miteinander erhält man bei einem Glanzwinkel von $\theta$ konstruktive Interferenz.
    Durch die Formel }
    2 \text{d} \sin \theta = \text{n} \lambda \label{7}
    \intertext{lässt sich bei einem Winkel $\theta$ und einer Gitterkonstante d, welche bei dem LiF-Kristall bei $d = 201,4\,\unit{\pico\meter}$ ist, die Wellenlänge $\lambda$ bestimmen.}
    \text{E} = \frac{\text{h} \cdot \nu}{2\text{d}\sin(\theta)} \label{8}
\end{align}
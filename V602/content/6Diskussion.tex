\section{Diskussion}

\begin{align*}
    \intertext{Im ersten Teil der Versuchsreihe wurde die Bragg-Bedingung untersucht.
    Dies führt zu einer relativen Abweichung von:}
    \theta_{\text{Exp}} = 27,21\unit{\degree} \,\,\,\,\,\,\,\,\,\, \theta_{\text{Theo}} = 28\unit{\degree} \\
    \to 2,82\%\,.
    \intertext{Die Abweichung ist sehr gering dennoch besteht eine gewisse Ungenauigkeit, die womöglich auf das Messen der Apparatur führt.
    Auffällig im zweiten Teil der Reihe sind folgende Abweichungen:}
\end{align*}

\begin{table}[H]
    \centering
    \caption{Abweichung der maximalen Energie mit $\text{E}_{\text{Theo}}$}
    \label{Tabelle3}
    \begin{tabular} {c c}
        \toprule
        {$ $} &
        {$ \text{E}_{\text{Theo}} = 35\text{keV}$} \\
        \midrule
        $\text{E}_{1}$ & 36,8\% \\
        $\text{E}_{2}$ & 27,4\% \\
        \bottomrule
    \end{tabular} 
\end{table}

\begin{table}[H]
    \centering
    \caption{Abweichung der Energien und Abschirmkonstanten} 
    \label{Tabelle4}
    \begin{tabular} {c | c  |c}
        \toprule
        {$ $} &
        {$\sigma_\text{Lit}$ mit $\sigma_\text{Exp}$ in \% } &
        {$\text{E}_\text{Lit}$ mit $\text{E}_\text{Exp}$ in \%} \\
        \midrule
        Zn & 1,40  & 1,13 \\
        Ga & 6,15  & 1,73 \\
        Br & 8,45  & 2,07 \\
        Sr & 14,25 & 3,06 \\
        Zr & 11,23 & 2,39 \\
        \bottomrule
    \end{tabular} 
\end{table}

\begin{flushleft}
Die gemessenen Energien weisen auf eine minimale Abweichung auf, dagegen sind die Abschirmkonstanten deutlich höher. 
Der höchste Wert beträgt $14,25\%$.
Die Ungenauigkeit führt auf das Ablesen der Werte oder auf eine Fehlerquelle der Apparatur selber, wie das Erfassen der Daten.
Es ist anzumerken, dass der Rechner um ein älteres Modell handelt.
Die mögliche Fehlerquelle lässt sich ebenso mit der maximalen Energie begründen, da die Abweichung deutlich größer ist als die anderen Abweichungen.
Zudem wurden mehrere Materialien getestet, da es häufig zu einer Unstimmigkeiten der Werte kam.
\end{flushleft}

\begin{align*}
    \intertext{Die Rydberg Konstante ließ sich aus der Steigung der Gerade bestimmen.
    Somit folgt für die Abweichung}
    \text{R}_{\infty} = (1,27 \pm 0,48) \cdot 10^{7}\, \frac{1}{\text{m}} \,\,\,\,\,\, \text{R}_{\infty, \text{Theo}} = 1,097 \cdot 10^{7}\, \frac{1}{\text{m}} \\
    \to 15,7\%
    \intertext{Grundsätzlich eignen sich alle Methoden, da sich die Messergebnisse in einem annehmbaren Bereich befinden, trotz der kleinen Ungenauigkeiten.}
\end{align*}

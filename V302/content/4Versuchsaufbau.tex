\section{Versuchsaufbau und Versuchdurchführung}

\subsection{Wheatstonesche Brücke}

\begin{flushleft}
    Die Schaltung wird wie in Abbildung \ref{Abbildung4} aufgebaut.
    Das Potentiometer wird so eingestellt, dass die Brückenspannung verschwindet.
    Wenn das Potentiometer so variiert wurde, dass die Brückenspannung verschwindet, werden die Werte für $R_{3}$ und $R_{4}$ notiert.
    $ R_{2} $ ist ein fest gewählter Widerstand und $R_{x}$ wird durch die Formel (\ref{3}) berechnet.
    Dieser Durchgang wird für drei verschiedene $ R_{2} $ Widerstände wiederholt, für jeweils zwei verschiedene unbekannte $R_{x}$ Widerstände.
\end{flushleft}

\subsection{Kapazitätsmessbrücke}

\begin{flushleft}
    Die Schaltung wird wie in Abbildung \ref{Abbildung5} aufgebaut.
    Man stellt das Potentiometer $R_{2}$ und $R_{3}$ wieder so ein, dass die Brückenspannung verschwindet und notiert die Werte für $R_{2}$, $R_{3}$ und $R_{4}$.
    Der Wert für $C_{2}$ ist fest gewählt, $R_{x}$ wird durch die Formel (\ref{4}) und $C_{x}$ durch die Formel (\ref{5}) bestimmt.
    Dies wird einmal wiederholt, danach werden die Widerstände $R_{x}$ und $R_{2}$ ausgebaut und der Versuch wird ein weiteres mal ausgeführt und es werden die Messwerte für $R_{3}$ und $R_{4}$ notiert.
\end{flushleft}

\subsection{Induktivitätsmessbrücke}

\begin{flushleft}
    Die Schaltung wird wie in Abbildung \ref{Abbildung6} aufgebaut und die Potentiometer $R_{2}$ und $ R_{3} $ werden erneut so eingestellt, dass die Brückenspannung verschwindet.
    Die Werte die aufgenommen werden sind wieder $R_{2}$, $R_{3}$ und $R_{4}$.
    Der Wert für $L_{2}$ ist fest gewählt, $R_{x}$ wird durch die Formel (\ref{4}) und $L_{x}$ durch die Formel (\ref{6}) bestimmt.
    Dies wird nur einmal durchgeführt.
\end{flushleft}

\subsection{Maxwell-Brücke}

\begin{flushleft}
    Die Schaltung wird wie in Abbildung \ref{Abbildung7} aufgebaut und erneut werden wie vorher die Potentiometer, welche diesmal $R_{3}$ und $R_{4}$ sind, so eingestellt, dass die Brückenspannung verschwindet.
    Der Widerstand $R_{2} $ ist diesmal fest gewählt, genauso wie der Kondensator $C_{4}$ und die Werte für $R_{3}$ und $R_{4}$ werden notiert.
    Der Widerstand $R_{x}$ wird über die Formel (\ref{7}) und die Spule $L_{x}$ über die Formel (\ref{8}) berechnet.
    Dadurch das die Maxwell-Brücke eine weitere Induktivitätsbrücke ist, ist der Wert für die Spule $L_{x}$ genauer als der Wert für $L_{x}$ von der Induktivitätsmessbrücke.
\end{flushleft}

\subsection{Wien-Robinson-Brücke}

\begin{flushleft}
    Die Schaltung wird wie in Abbildung \ref{Abbildung8} aufgebaut.
    Hierbei wird kein Bauteil bestimmt oder variiert, sondern die Brückenspannung für verschiedene Frequenzen ermittelt.
    Die Frequenzen werden im Bereich von $ 20\,\unit{\hertz} \leq v \leq 30000\,\unit{\hertz} $ varriert und die Spannungen $U_{\text{Br}}$ aufgenommen und notiert. 
\end{flushleft}

\section{Diskussion}

\begin{flushleft}
    Werden die einzelnen Bauteile in Betracht gezogen,so fällt auf, dass die meisten Abweichungen von den Bauteilen ausgehen.
    Dabei sind die Abweichungen gering, was dazu führt, dass die Bestimmung der gesuchten Werte nahezu gleich sind.
    Wie bei der Wheatstoneschen Brücke, besitzt die bestimmte Größe eine relative Abweichung von $-1,56\%$ vom Literaturwert $R_{\text{x,10}} = 239\,\unit{\ohm}$.
    Für den Wert 13 beträgt die Abweichung $-0,3\%$, welcher sich im Toleranzbereich befindet. 
    Bei der Kapazitätmessbrücke, welche eine Eichgenauigkeit von $3\%$ laut Hersteller besitzt \cite{a1}, hat eine relative Abweichung von $-1,22\%$.
    Auffällig ist, dass der bei der Wheatstoneschen Brücke sowie der Kapazitätmessbrücke, die jeweiligen Abweichungen größtenteils sich im Toleranzbereich befindet.
    Beim  Vergleich der Messung mit der Induktivitätsbrücke sowie der Maxwell-Brücke, so fällt auf, dass die Messwerte sich deutlich vom Literaturwert $L_{18} = 49,82\,\unit{\milli\henry}$ und $R_{18} = 360,5\,\unit{\ohm}$ unterscheiden.
    Bei der Induktivitätsmessbrücke beträgt die relative Abweichung der Induktivität $33,48\%$ und der Widerstand $13,88\%$.
    Bei der Maxwell-Brücke ist die Abweichung deutlich kleiner, mit $8,99\%$ bei der Induktivität und $-10\%$ bei der Bestimmung des Innenwiderstandes. 
    Der Unterschied lässt sich darauf hinweisen, dass die Induktivitätsmessbrücke keinem Innenwiderstand besitzt, wodurch sich dann die Verluste durch auftretende Wärmeenergie vorweisen lässt.
    Dabei besitzt die Maxwell-Brücke einen eingebauten Kondensator $C_{4}$, wodurch sich das Ergebnis deutlich näher am Toleranzbereich befindet. 
    Dennoch treten kleine Abweichungen auf, die aber auf das menschliche Geschick zurückzuführen sind. 
    Bei Beobachtung sowie das präzise Einstellen des Reglers, bis zum Verschwinden der Brückenspannung, weist ebenso auf eine mögliche Abweichung.
    Die Vermutung ,dass eine Brückenspannung bei $v_{0}$ trotzdem durch Oberwellen entsteht, wurde durch den Klirrfaktor bestätigt.
\end{flushleft}
\section{Zielsetzung} 

\begin{flushleft}
    Im Versuch V601 geht es um die Bestimmung der diskreten Energiewerte der Elektronenhülle eines Atoms.
    Die Energie kann dabei aus dem Elektronenstoßexperiment, also aus elastischen und inelastischen Stößen zwischen den Elektronen und den Hg-Atomen, berechnet werden.
    Dafür wird die Energiedifferenz zwischen vor und dem nach dem Stoß genutzt, welche in diesem Versuch bestimmt wird.
    Ebenso wird die Energieverteilung und die Ionisierungsenergie von Quecksilber bestimmt.
\end{flushleft}
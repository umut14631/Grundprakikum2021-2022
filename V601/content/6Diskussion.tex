\section{Diskussion}

\begin{flushleft}
    Ein Vergleich der beiden vom XY-Schreiber erstellten Graphen in Abschnitt \ref{sec:2} und der daraus resultierenden Energieverteilung zeigt, dass bei der Messung bei Raumtemperatur eine deutliche Abweichung von der Theoriekurve ersichtlich ist.
    Theoretisch sollte eine Gauß-Kurve zu sehen sein, jedoch ist die Steigung des Graphen bei geringer Bremsspannung auffällig stark.
    Zudem reichte die Messung an dem Schreiber nicht, welche wiederum auf die nicht ausreichende Justierung der Regler führt.
\end{flushleft}

\begin{flushleft}
    Deutlich ausgeprägter sind die Maxima der Franck-Hertz-Kurve bei der Messung \ref{sec:Kap1}.
    Die mit Hilfe der Abstände bestimmte Anregungsenergie von Quecksilber zeigt hierbei folgende Abweichung
\end{flushleft}

\begin{table}[H]
    \centering
    \caption{Abweichungen der Anregungsenergien.} 
    \label{Tabelle4}
    \begin{tabular} {c | c  c  c}
        \toprule
        {$ $} &
        {$ \text{E}_{\text{Lit}} = 4,9\,\text{eV}\,\text{\cite{a2}} \,\,\,\text{in \%} $} \\
        \midrule
        $\increment \text{E}_{\text{B}}$ & 7,5 \\
        $\increment \text{E}_{\text{S}}$ & 5,7 \\
        \bottomrule
    \end{tabular} 
\end{table}

\begin{flushleft}
    Auch dem erneut bestimmten Kontaktpotential stellt folgende Abweichung zudem in Messung \ref{sec:Kap0} dar.
\end{flushleft}

\begin{table}[H]
    \centering
    \caption{Abweichungen der Kontaktpotentiale.} 
    \label{Tabelle4}
    \begin{tabular} {c | c  c  c}
        \toprule
        {$ $} &
        {$ \text{K} = 0,5\,\text{eV} \,\,\,\text{in \%} $} \\
        \midrule
        $\increment \text{K}_{\text{B}}$ & 28 \\
        $\increment \text{K}_{\text{S}}$ & 2 \\
        \bottomrule
    \end{tabular} 
\end{table}

\begin{flushleft}
    Obwohl nur geringe Abweichungen vorhanden sind existieren dennoch gewisse Fehlerquellen wie das falsche Ablesen am XY-Schreiber oder am Spannungsgerät.
    Die nächste mögliche Fehlerquelle ist die Regulierung der Temperatur.
    Aufgrund äußerer Gegebenheiten kam es zu Temperaturschwankungen,  welche zwar durch konstantes Regulieren der Temperatur gering gehalten werden sollte, jedoch zusätzliche Auswirkungen auf die Kurve hatten.
\end{flushleft}

\begin{flushleft}
    Nichtsdestotrotz ist der Versuch ein Beweis für die Quantisierbarkeit der Energieniveaus.
\end{flushleft}
\newpage
\section{Diskussion}

\begin{align*}
    \intertext{Im ersten Teil des Versuches sind die Messpaare im Toleranzbereich, dies wird durch die jeweiligen Graphen, Abbildung \ref{Abbildung5}, verdeutlicht.
    Die Messwerte liegen ebenso auf der Theoriekurve, mit Ausnahme von einigen Werten.
    Verglichen mit dem Literaturwert \cite{a2} und dem ausgerechneten Wert, folgt für die relative Abweichung}
    L_{\text{gemessen}} = ( 17.79 \pm 0,083 ) \cdot 10^{3} \frac{\unit{\joule}}{\text{mol}}\\
    L_{\text{Wasser,Lit}} = 40.8 \cdot 10^{3} \frac{\unit{\joule}}{\text{mol}} \\
    \quad \to  \text{relative Abweichung} = 55.5\%\,.
    \intertext{Wie zu erkennen ist der gemessene Wert halb so groß wie der Literaturwert.
    Die Abweichung würde sich durch das Ablesen der Temperaturen zurückführen.
    Zudem wurde die Temperatur nicht konstant gemessen, sondern immer wieder durch das Regulieren der Wasserpumpe beeinflusst.
    Dies führte dazu, dass der Druck ab und zu auf dem Messgerät schwankte.
    Im zweiten Teil des Versuches ist es auffällig, dass die Messwerte nah an der Ausgleichskurve liegen.
    Am Anfang des ersten Falles ist ein Abstand der Temperaturen zu erkennen, wodurch sich ein hoher Anstieg der Temperatur bemerkbar macht. 
    Da es sich um ein abgeschlossenen System handelt, wird die Apparatur schnell erhitzt.}
\end{align*}

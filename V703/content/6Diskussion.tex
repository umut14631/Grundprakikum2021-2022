\section{Diskussion}

\begin{flushleft}
    Durch die gemessenen Werte formt sich die Kurve der Charakteristik des Zählrohres, dennoch ist sie nicht vollständig ausgeprägt.
    Mit einer Plateausteigung von $\text{m} = 0,978\%$ pro $1000\,\unit{\volt}$ ist die Steigung akzeptabel, denn je näher an einer Steigung von $0\%$, desto besser der Bereich.
\end{flushleft}

\begin{align*}
    \intertext{Die Abweichung der gemessenen Totzeit und der errechneten Totzeit beträgt}
    \text{T}_{\text{Osz.}} = (80 \pm 5)\,\unit{\micro\second} \,\,\,\,\,\, \text{T} = (72,29 \pm 4,41)\,\unit{\micro\second} \\
    \to 10,66\%\,.
\end{align*}

\begin{flushleft}
    Obwohl die Abweichung gering ist sind Ungenauigkeiten vorhanden.
    Mögliche Fehlerquellen sind z.B das falsche Ablesen der Totzeit oder das viele Auftreten der Nachentladungen. 
    Die Beziehung der Spannung und der freigesetzten Ladungsmenge, ein linearer Anstieg, wurde mit der Regression bekräftigt.
    Resümierend lässt sich sagen, dass durch den Versuch am Geiger-Mülle-Zählrohr die Funktionsweise der Apparatur sowie die verschiedenen Kenngrößen angelernt und untersucht werden können.
\end{flushleft}
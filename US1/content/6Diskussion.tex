\section{Diskussion}

\begin{flushleft}
    Die Ermittlung der Schallgeschwindigkeit mit der Geräteeinstellung, des Impuls-Echo-Verfahren sowie des Durchschallungs-Verfahrens weisen folgende Abweichungen auf.
\end{flushleft}

\begin{table}[H]
    \centering
    \begin{tabular} {c | c}
        \toprule
        { } &
        {Abweichung zu $ \text{c}_{\text{acryl}} = 2730\,\frac{\unit{\meter}}{\unit{\second}} $ \cite{a1} in \%} \\
        \midrule
        $\text{c}_{\text{exp}}   = 2777,7\,\frac{\unit{\meter}}{\unit{\second}} $ & 1,74 \\
        $ \text{A}_{\text{Echo}}   = 2723\,\frac{\unit{\meter}}{\unit{\second}} $ & 0,25 \\
        $ \text{A}_{\text{Durch.}} = 2663\,\frac{\unit{\meter}}{\unit{\second}} $ & 2,45 \\
        \bottomrule
    \end{tabular} 
\end{table}

\begin{flushleft}
    Die Abweichungen sind sehr gering somit erweist sich die Bestimmung der Schallgeschwindigkeit mit dem Impuls-Echo-Verfahren am profitabelsten.
\end{flushleft}

\begin{flushleft}
    Die Dicke der Scheiben haben eine Abweichung von
\end{flushleft}

\begin{table}[H]
    \centering
    \begin{tabular} {c |  c  c}
        \toprule
        { } &
        {$ \text{Scheibe}_{\text{Lit.}}\,(\text{o.})$ in \% }  &
        {$ \text{Scheibe}_{\text{Lit.}}\,(\text{u.}) $ in \%}  \\
        \midrule
        $\text{Scheibe}\,(\text{o.}) $ & 3,8 & - \\
        $\text{Scheibe}\,(\text{u.}) $  & -  & 3,3  \\
        \bottomrule
    \end{tabular} 
\end{table}

\begin{flushleft}
    Für das Abmessung der Auge lassen sich Referenzwerk finden, die ungefähr den berechneten Abständen entsprechen.
    Dennoch weisen bei der Versuchsreihe gewisse Ungenauigkeiten bzw. Abweichungen auf.
    Die Fehlerquellen liegen zum einen bei der begrenzten Ablesemöglichkeit oder die falsche Untersuchungen an den Grafiken.
    Die Ultraschallsonden sind ebenso nicht präzise bei der Messung und können mit einer begrenzten Genauigkeit die Amplituden und Laufzeiten messen.
\end{flushleft}

\begin{flushleft}
    Nichtsdestotrotz eignet sich das Verfahren zur Untersuchung von Materialien mit verschiedenen Ultraschallmessmethoden.
\end{flushleft}
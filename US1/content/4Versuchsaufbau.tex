\section{Versuchsaufbau und Versuchsdurchführung}

\begin{flushleft}
    Die benötigten Materialien für den Versuch setzen sich zusammen aus einem Echoskop, mit zwei 2 MHz Sonden, einem Rechner mit dem Programm Echoview, vier verschieden großen Acrylzylindern, zwei verschieden dicken Acrylplatten, Kontaktgel, bidestiliertes Wasser, einer Schieblehre, Halterungen für die Ultraschallsonden und Küchentücher. 
    Der Grundaufbau besteht aus dem Echoskop welches mit dem Rechner verbunden ist. 
    Auf diesem Rechner wird das Programm Echoview gestartet und verwendet. 
\end{flushleft}

\subsection{Programm und Geräteeinstellung}

\subsubsection{Teil 1}

\begin{flushleft}
    Der Grundaufbau wird mit einer der beiden Sonden verwendet und eine des beiden Acrylplatten untersucht.
    Die Acrylplatte wird horizontal auf ein Küchentuch gelegt und mit bidestiliertem Wasser angefeuchtet. 
    Danach wird die Sonde auf die Acrylplatte gelegt und mit dem Programm Echoview ein A-Scan durchgeführt.
    Wichtig hierbei ist das die Verstärkung so eingestellt wird, dass mindestens vier Reflexe gut sichtbar sind.
    Aufgenommen werden für alle Reflexe die Laufzeiten sowie die Amplitude.
    Als letztes wird die dicke der Acrylplatte mit einer Schieblehre vermessen und mithilfe der Laufzeiten die Schallgeschwindigkeit ausgerechnet.
\end{flushleft}

\subsubsection{Teil 2}

\begin{flushleft}
    In dem zweiten Teils kann zwischen drei möglichen Programmeinstellungen gewechselt werden, AM, HF und AM + Hf.
    Verwendet wird für diesen Teil die Einstellung AM + HF, sowie die zuvor errechnete Schallgeschwindigkeit.
    Der Versuchsaufbau bleibt identisch mit dem vom Versuchsteil 1. 
    Im folgenden werden für fünf Schwingungen die Periode gemessen und dadurch die gemittelte Frequenz, sowie die gemittelte Wellenlänge berechnet und mit den Theoriewerten verglichen.
\end{flushleft}

\subsubsection{Teil 3}

\begin{flushleft}
    Der Aufbau wird nicht verändert.
    Als letztes wird eine Tiefenmessung durchgeführt wodurch die Dicke der Acrylplatte bestimmt werden soll.
    Diese wird verglichen mit der zuvor gemessenen Dicke. 
\end{flushleft}

\subsection{Bestimmung der Schallgeschwindigkeit und Dämpfung mit dem Impuls-Echo-Verfahren}

\begin{flushleft} 
    Der Grundaufbau wird erneut verwendet.
    Einer der vier Acrylzylinder wird auf ein Küchenrollentuch gestellt und mit bidestiliertem Wasser wird eine Sonde an die Glatte Oberfläche gekoppelt.
    Gemessen wird die Laufzeit und Amplitude der Welle .
    Dies wird für alle vier Zylinder wiederholt.
\end{flushleft}

\subsection{Bestimmung der Schallgeschwindigkeit mit dem Durchschallungs-Verfahren}

\begin{flushleft}
    Der Grundaufbau wird erneut verwendet.
    Einer der vier Acrylzylinder wird in eine Halterung gesetzt, sodass der Zylinder horizontal positioniert ist.
    Beide Sonden werden hierbei verwendet, da eine als Sender und die andere als Empfänger dient.
    Beide Sonden werden in eine Halterung gesetzt und mit Koppelgel an die glatten Oberflächen gekoppelt.
    Gemessen werden erneut die Laufzeiten für alle vier Zylinder. 
\end{flushleft}

\subsection{Abbildung des Spektrums und des Cepstrum}

\begin{flushleft}
    Der Grundaufbau wird erneut verwendet.
    Es werden beide Acrylplatten und ein ca. $40\,\unit{\milli\meter}$ hoher Acrylzylinder aufeinander gestellt.
    Wichtig hierbei ist, dass die einzelnen Acrylelemente mit bidestiliertem gekoppelt werden. 
    Eine der beiden Sonden wird mit bidestiliertem Wasser auf der glatten Zylinderoberfläche gekoppelt. 
    Der Zylinder dient dabei als Vorlaufstrecke, damit die Mehrfachechos besser von dem Initialecho getrennt werden können.
    Die Verstärkung wird so eingestellt, dass möglichst drei Mehrfachreflexionen zusehen sind.
    Als letztes wird das Programm so eingestellt, dass ein Spektrum und ein Cepstrum zu sehen ist.
    Dies wird als Bild exportiert und interpretiert.
\end{flushleft}

\subsection{Untersuchung eines Augenmodels}

\begin{flushleft}
    Bei der Untersuchung des Augenmodells  wird eine der Grundaufbau verwendet.
    Diese wird mit dem zu untersuchendem Augenmodell erweitert.
    Es wird eine $2\,\unit{\mega\hertz}$ Sonde mit Koppelgel auf die Hornhaut gekoppelt.
    Die Ultraschallsonde wird mit leichtem Druck an die Hornhaut gedrückt und der Winkel, in dem die Sonde gehalten wird, solange verändert bis Echo an der Rückwand  der Retina zusehen ist.
    Bestimmt werden die Laufzeiten der einzelnen Schichten und die Graphik wird exportiert.
\end{flushleft}
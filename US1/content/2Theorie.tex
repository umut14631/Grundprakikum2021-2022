\section{Theorie}

\begin{align}
    \intertext{Ultraschallwellen, sind  longitudinale Wellen, haben eine Frequenz von $20\,\unit{\kilo\hertz}$ bis $1\unit{\giga\hertz}$ und werden in der Medizin für die zerstörungsfreie Werkstoffprüfung verwendet.}
    p(x,t) = p_{0} +v_{0}\,\text{Z}\,\cos(wt\,-\,kx) \label{1}
    \intertext{Die Ultraschallwelle bewegt sich durch Druckschwankungen fort, wobei $Z = c \cdot \rho$ die akustische Impedanz ist.
    Diese setzt sich aus der Schallgeschwindigkeit $c$ und der Dichte $\rho$ zusammen.
    Der Unterschied zwischen einer Ultraschallwelle und einer elektromagnetischen Welle liegt bei der Phasengeschwindigkeit, aufgrund der materialabhängigkeit die durch Druck bzw. Dichteänderung verändert wird.
    Die Schallwellen breiten sich als Longitudinalwelle in Flüssigkeit und Gasen aus wobei die Geschwindigkeit, in Flüssigkeit, dieser von der Kompressibilität sowie Dichte abhängt}
    c_{Fl} = \sqrt{\frac{1}{\kappa \cdot \rho}}. \label{2}
    \intertext{Bei Festkörpern breitet sich die Welle als Longitudinalwelle oder Transversalwelle aus, mit dem Elastizitätsmodul $E$, was für die Kompressibilität $\frac{1}{\kappa}$ ersetzt wird}
    c_{Fe} = \sqrt{\frac{E}{\rho}}. \label{3}
    \intertext{Die Schallgeschwindigkeit der beiden Wellenarten unterscheidet sich voneinander, wobei die Schallgeschwindigkeit in Festkörpern Richtungsabhängig.
    Bei der Schallausbreitung wird ein Teil der Energie absorbiert und die Intensität $I_{0}$ fällt nach der Strecke $x$ exponentiell ab.}
    I(x) = I_{0} \cdot e^{\alpha x} \label{4}
    \intertext{\alpha ist hierbei der Absorptionskoeffizient der Schallamplitude.}
    \intertext{Bei dem Auftreffen einer Schallwelle auf eine Grenzfläche wird ein Teil reflektiert und der Reflektionskoeffizient $R$ setzt sich aus der akustischen Impedanz des auftreffmaterials zusammen.}
    R = \left(\frac{Z_{1} - Z_{2}}{Z_{1} + Z_{2}}\right)^{2}. \label{5}
    \intertext{Die Transmission $T$ lässt sich durch $T = 1 - R$ berechnen.}
    \intertext{Um Ultraschall zu erzeugen gibt es verschiedene Methoden.
    Eine davon ist durch die Anwendung des reziproken piezo-elektrischen Effekt.
    Dabei wird ein Kristall in ein Wechselfeld gebracht, welches den Kristall zum schwingen bringt, was wiederum die Ultraschallwellen erzeugt.
    Wenn die Eigenfrequenz mit der Anregungsfrequenz übereinstimmt können große Schwingungsamplituden erzeugt werden.
    Die daraus resultierenden Schallenergien sind extrem hoch und können genutzt werden.
    Der Piezokristall kann nicht nur als Erzeuger sondern auch als Empfänger für Ultraschallwellen genutzt werden.
    Ultraschallwellen werden in der Medizin dafür verwendet, Informationen über den durchstrahlten Körper zu erhalten.
    Die Laufzeitmessung ist ein häufig verwendetes Messverfahren, wobei kurzzeitige Impulse in Richtung Empfänger ausgesendet werden, womit die Strecke zum Empfänger bestimmt werden kann.} \notag
\end{align}

\begin{align}
    \intertext{Das Durchschallungs-Verfahren ist eines der zwei Methoden die bei diesem Verfahren angewandt wird.
    Bei diesem Verfahren werden kurzzeitige Schallimpulse ausgesendet, welche auf der anderen Seite wieder von einem Ultraschallempfänger aufgefangen wird.
    Falls eine Fehlstelle auf diesem Weg vorhanden ist, wird eine abgeschwächtere Intensität gemessen. 
    Jedoch kann nicht gesagt werden wo sich diese befindet.
    Bei dem Impuls-Echo-Verfahren ist die Sonde Ultraschallsender sowie empfänger.
    Reflekitert wird die Ultraschallwelle hierbei an der Grenzfläche des durchstrahlten Objekts.
    Bei vorhandenen Fehlstellen gibt die Größe des Signals Rückschlüssse zu der Größe der Fehlstelle.
    Über die Laufzeit kann die Lage der Fehlstelle ermittelt werden.}
    \text{s} = \frac{1}{2} c \text{t} \label{6}
    \intertext{Die Laufzeit kann in einem A-Scan, B-Scan oder TM-Scan als Diagramm dargestellt werden.} \notag
\end{align}
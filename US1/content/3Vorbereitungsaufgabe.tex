\section{Vorbereitungsaufgabe}

\begin{flushleft}
    Bei der ersten Vorbereitungsaufgabe sollen die Literaturwerte der Schallgeschwindigkeit, sowie der akustischen Impedanz, für verschiedenen Medien recherchiert werden.
\end{flushleft}

\begin{table}[H]
    \centering
    \caption{Die einzelnen Werte zu der ersten Vorbereitungsaufgabe} 
    \label{Tabelle1}
    \begin{tabular} {c  c  c}
        \toprule
        {$  $} &
        {$ \text{c} \mathbin{/} \frac{\unit{\meter}}{\unit{\second}} $} &
        {$ \text{Z} = \text{c}\cdot\rho \mathbin{/} 10^{6}\,\frac{\unit{\kilo\gram}}{\unit{\second}\,\unit{\meter}^2} $} \\
        \midrule
        Luft                & 331 \cite{a1} & $43 \cdot 10^{-5}$ \cite{a1} \\
        destiliertes Wasser & 1492 \cite{a1}& 1,48 \cite{a1} \\
        Blut                & 1530 \cite{a1} & 1,63 \cite{a1} \\
        Knochen             & 3600 \cite{a1} & 1,70 \cite{a1} \\
        Acryl               & 2730 \cite{a3} &  \\
        \bottomrule
    \end{tabular} 
\end{table}

\begin{align*}
    \intertext{Bei der zweiten Vorbereitungsaufgabe sollen die Wellenlängen und die Periode einer $2\,\unit{\mega\hertz}$ Schwingung in Acryl berechnet werden. Dies geschieht wie folgt}
    \text{T} = \frac{1}{\text{f}} = \frac{1}{2\cdot 10^{6}}\,\frac{1}{\unit{\second}}\,, \\
    \lambda = \frac{\text{c}_{\text{Acryl}}}{\text{f}} = 1,365 \cdot 10^{-3}\,\unit{\meter}\,.
\end{align*}

\section{Diskussion}

\begin{flushleft}
    Auffällig bei der Bestimmung der Elementarladung sind folgende Verhältnisse mit dem Literaturwert \cite{a3}.
\end{flushleft}

\begin{table}[H]
    \centering
    \caption{Abweichung der Ladung.}
    \label{Tabelle4}
    \begin{tabular} {c  c}
        \toprule
        {} &
        {Abweichung mit $e_{\text{Lit}} = 1,602 \cdot 10^{-19}\,\unit{\coulomb}$ in \%} \\
        \midrule
        $e_{0, \text{unkorrigiert}}$ & 9,17\\
        $e_{0, \text{korrigiert}}$ & 3,05\\
        \bottomrule
    \end{tabular} 
\end{table}

\begin{flushleft}
    Beide Methoden eignen sich für die Bestimmung der Elementarladung.
    Die korrigierte Version erweist sich als profitabler, da die Abweichung sehr gering ist.
    Jedoch ist es anzumerken, dass mehr als die Hälfte der Messungen nicht im vorgenommenen Intervall lagen, die auf mögliche Fehlerquellen hinweisen.
    Zudem musste das Ablesen, sowie das stoppen der Zeit gleichzeitig passieren, um Messungenauigkeiten zu minimieren.
    Trotzdem könnten Absetzer entstanden sein, wodurch die Zeiten unpassender geworden sind.
    Ebenso könnten sich die Ladungen der Tröpfchen während des Prozesses verändert haben.
    Nichtsdestotrotz liefern die vorhandenen Messdaten akzeptable Ergebnisse. 
    Für die daraus resultierende Avogadrokonstante:
\end{flushleft}

\begin{table}[H]
    \centering
    \caption{Abweichung der Konstanten.}
    \label{Tabelle5}
    \begin{tabular} {c  c}
        \toprule
        {} &
        {Abweichung mit $\text{N}_{\text{a, Lit}} = 6,022 \cdot 10^{23}\,\frac{1}{\text{mol}}$ \cite{a2} in \%} \\
        \midrule
        $\text{N}_{\text{a, unkorrigiert}}$ & 8,4\\
        $\text{N}_{\text{a, korrigiert}}$   & 2,9\\
        \bottomrule
    \end{tabular} 
\end{table}

\begin{flushleft}
    Grundsätzlich eignen sich beide Methoden trotz der vielen Messunsicherheiten.
\end{flushleft}
\section{Diskussion}

\begin{align*}
    \intertext{Im ersten Teil der Versuchsreihe wurde eine Durchlasskurve erstellt, mit der daraus resultierenden Güte.
    In der Aufgabenstellung war eine Güte von $\text{Q} = 100$ angegeben. 
    Jedoch fiel die Messung mit einer Güte von $\text{Q} = 100$ schwer, da die Messapparatur die Ergebnisse nicht klar anzeigte.
    Aus diesem Anlass wurde eine Güte $\text{Q} = 20$ angenommen.
    Es war zu erwarten, dass der Bandpassfilter ein Maximum bei $\text{f} = 35\,\unit{\kilo\hertz}$ anzeigen würde, jedoch stellte sich fest, dass sich das Maximum bei $\text{f} = 22\,\unit{\kilo\hertz}$ befand.
    Die resultierende Abweichung lautet
    }
    \text{Q}_{\text{exp}} = (27,5 \pm 0,5) \,\,\,\,\,\, \text{Q}_{\text{exp}} = 20 \\
    \to 25\% \,.
    \intertext{Die Messung enthält viele Ungenauigkeiten. 
    Das Einstellen der Durchlassfrequenz schwankte schlagartig, wodurch klare Messpaare schwer zustande kamen. 
    Zeitlich wurde das Messgerät auf ein Oszilloskopen gewechselt, womit die Messung einfacher wurde.
    Dennoch schwankte der Graph durch die ständige Änderung der Frequenz.
    Somit entstanden mögliche Ablesefehler.}
\end{align*}

\begin{flushleft}
    Der zweite Teil des Versuches erfolgte brüchig.
    Der Sinusgenerator stürzte bei jeder Messung ab, wodurch sich der Zeiger ständig änderte und das Ablesen dadurch schwerer wurde.
    Die ersten Proben erfolgten trotzdem positiv, jedoch änderte sich die Messung bei der Probe $\text{Nd}_{2}\text{O}_{3}$ nicht. 
    Dessen ungeachtet folgten ausreichende Messergebnisse. 
    Auffällig sind hierbei die Abweichungen:
\end{flushleft}

\begin{table}[H]
    \centering
    \caption{Suszeptibilität der seltenen Erd-Elemente.} 
    \label{Tabelle8}
    \begin{tabular} {c |  c  c  c}
        \toprule
        {Ionen} &
        {$ \chi_{\text{U}}$ mit $\chi_{\text{T}}$ in \% } &
        {$ \chi_{\text{R}}$ mit $\chi_{\text{T}}$ in \% } \\
        \midrule
        $\text{Dy}_{2}\text{O}_{3}$ & 49,50 & 15,90 \\
        $\text{Gd}_{2}\text{O}_{3}$ & 65,40 & 26,59 \\
        $\text{Nd}_{2}\text{O}_{3}$ & 88,26 & 16,00 \\
        \bottomrule
    \end{tabular} 
\end{table}

\begin{flushleft}
    Wie zu erwarten ist die Abweichung von $\text{Nd}_{2}\text{O}_{3}$ sehr hoch.
    Die Messung mit der Probe erfolgte so gut wie gar nicht.
    Der Zeiger der Messapparatur bewegte sich kaum, weder vor dem Abgleich oder danach.
    Nichtsdestotrotz sind die Ergebnisse der anderen Proben, trotz der vielen Fehlerarten akzeptabel.
    Die Messung der Suszeptibilität durch die Widerstandsänderung $\increment \text{R}$ erwies sich profitabler als die Messung über die Brückenspannung.
\end{flushleft}

\begin{flushleft}
    Es ist anzumerken, dass die Apparatur einen 10-mal Verstärker hatte, allerdings hat die Verstärkung keinen großen Effekt auf die Messung, da sich nur die Größenordnung verändert.
    Sie diente für das bessere Ablesenden der Werte.
\end{flushleft}
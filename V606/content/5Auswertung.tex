\section{Auswertung} 

\subsection{Untersuchung des Selektiv-Verstärkers}

\begin{flushleft}
    Untersucht wird die Ausgangsspannung $\text{U}_{\text{A}}$ des Selektivverstärkers bei konstanter Eingangsspannung $\text{U}_{\text{E}}$ in Abhängigkeit von der Frequenz. 
    Durchnommen wird die Aufgabe mit einer Güte von $\text{Q} = 20$, sowie mit Durchlassfrequenzen zwischen $10\,\unit{\kilo\hertz}$ bis $30\,\unit{\kilo\hertz}$.
    Die Frequenzen $\nu$ mit den jeweiligen Ausgangsspannungen werden in Tabelle \ref{Tabelle1} festgehalten. 
    Aus der daraus resultierenden Durchlasskurve, Abbildung \ref{Abbildung6}, wird die Güte nach der Formel (\ref{12}) dazu wie in Abbildung \ref{Abbildung3} berechnet.
\end{flushleft}

\begin{table}[H]
    \centering
    \caption{Messwerte zur Untersuchung des Selektiv-Verstärkers.} 
    \label{Tabelle1}
    \begin{tabular} {c  c  c  c}
        \toprule
        {$ \text{f} \mathbin{/} \unit{\kilo\hertz} $} &
        {$ \text{U} \mathbin{/} \unit{\volt} $} &
        {$ \text{f} \mathbin{/} \unit{\kilo\hertz} $} &
        {$ \text{U} \mathbin{/} \unit{\volt} $} \\
        \midrule
        10,0 & 0,19 & 20,2 & 1,70 \\
        10,8 & 0,22 & 20,1 & 1,80 \\
        11,5 & 0,22 & 20,5 & 2,60 \\
        12,2 & 0,24 & 20,9 & 3,00 \\
        12,7 & 0,26 & 21,5 & 13,0 \\
        13,3 & 0,29 & 22,0 & 14,0 \\
        14,0 & 0,32 & 22,5 & 5,60 \\
        14,4 & 0,34 & 23,1 & 2,40 \\
        14,9 & 0,36 & 23,5 & 2,00 \\
        15,2 & 0,40 & 24,0 & 1,50 \\
        16,1 & 0,48 & 24,8 & 1,10 \\
        17,1 & 0,60 & 25,2 & 1,00 \\
        17,6 & 0,72 & 26,1 & 0,80 \\
        18,3 & 0,85 & 26,9 & 0,72 \\
        18,7 & 0,95 & 27,6 & 0,60 \\
        19,2 & 1,15 & 29,3 & 0,48 \\
        19,5 & 1,30 & 29,9 & 0,44 \\
        \bottomrule
    \end{tabular} 
\end{table}

\begin{figure}[H]
    \centering
    \includegraphics[height=80mm]{bilder/Gaußglocke.png}
    \caption{Durchlasskurve des Selektiv-Verstärkers\label{Abbildung6} }
\end{figure}

\begin{align*}
    \intertext{Daraus folgt für die Güte}
    \text{Q} = ( 27,5 \pm 0,5 )\,.
\end{align*}

\subsection{Unter der Suszeptibilität seltener Erd-Elemente}

\begin{align}
    \intertext{Für den Aufgabenteil wird eine Spule mit der Länge $\text{l} = 0,135\,\unit{\meter}$, Widerstand $\text{R} = 0,7\,\unit{\ohm}$, Windungszahl $\text{n} = 250$ und einer Querschnittsfläche von $\text{F} = 86,6 \cdot 10^{-6}\,\unit{\meter}^2$ verwendet.
    Da die Proben aus staubförmigen Material bestehen und sich nicht beliebig stopfen lassen ist ihre Dichte geringer.
    Aus diesem Grund wird die Querschnittsfläche nach der Formel}
    \text{Q}_{\text{real}} = \frac{\text{m}_{\text{p}}}{\text{L} \cdot \delta_{\text{w}}}\,, \label{13}
    \intertext{für jede einzelne Probe korrigiert.
    Benutzt werden drei Ionen mit den jeweiligen Daten in der Tabelle \ref{Tabelle2}.} \notag
\end{align}


\begin{table}[H]     
    \centering
    \caption{Daten zu dem Ionen.} 
    \label{Tabelle2}
    \begin{tabular} {c  c  c  c  c}
        \toprule
        {Ionen} &
        {$ \text{m}_{\text{p}} \mathbin{/} \si{\kilogram}$} &
        {$ \text{L} \mathbin{/} 10^{-2}\,\unit{\meter}$} & 
        {$ \delta_{\text{w}} \mathbin{/} \frac{\si{\kilogram}}{\unit{\meter}^{3}}$} &
        {$ \text{Q}_{\text{real}} \mathbin{/} 10^{-6}\,\unit{\meter}^2$} \\
        \midrule
        $\text{Dy}_{2}\text{O}_{3}$ & 0,0151 & $15,40 \pm 0,10$ & 7800 & 12,57\\
        $\text{Gd}_{2}\text{O}_{3}$ & 0,0141 & $15,73 \pm 0,15$ & 7400 & 12,11\\
        $\text{Nd}_{2}\text{O}_{3}$ & 0,0185 & $15,56 \pm 0,21$ & 7240 & 16,42\\
    \end{tabular} 
\end{table}

\begin{flushleft}
    Gemessen wird die Brückenspannung sowie der Widerstand nachdem Abgleich ohne Probe.
    Die Probe wird eingeführt, die Brückenspannung wird abgelesen und nachdem Abgleich wird der Widerstand abgelesen.
    Der Vorgang wiederholt sich für alle drei Ionen und wird in Tabelle \ref{Tabelle3} festgehalten. 
    Aus der Widerstandsänderung $\increment\text{R}$, der Formel (\ref{10}) und (\ref{11}) resultiert die Größe der Suszeptibilität, notiert in Tabelle \ref{Tabelle5}. 
    Der Betrag der Speisespannung beiträgt $\lvert \text{U}_{\text{sp}} \rvert = 1\,\unit{\volt}$ und $\text{R}_{3}$ beträgt $1000\,\unit{\ohm}$.
\end{flushleft}

\begin{table}[H]
    \centering
    \caption{Messergebnisse der Ionen.} 
    \label{Tabelle3}
    \begin{tabular} {c || c  c  c  c  c  c}
        \toprule
        {Ionen} &
        {$ \text{U}_{\text{oP}} \mathbin{/} 10^{-3}\,\unit{\volt} $} &
        {$ \text{R}_{\text{oP}} \mathbin{/} 10^{-3}\,\unit{\ohm} $} &
        {$ \text{U}_{\text{mP}} \mathbin{/} 10^{-3}\,\unit{\volt} $} &
        {$ \text{R}_{\text{mP}} \mathbin{/} 10^{-3}\,\unit{\ohm} $} &
        {$ \increment\text{U} \mathbin{/} 10^{-3}\,\unit{\volt} $} &
        {$ \increment\text{R} \mathbin{/} 10^{-3}\,\unit{\ohm} $}  \\
        \midrule
        \underline{$\text{Dy}_{2}\text{O}_{3}$} & 13,0 & 2525 & 17,5 & 990  & 4,5 & 1535 \\
                                                & 12,5 & 2370 & 15,5 & 1050 & 3,0 & 1320 \\
                                                & 12,0 & 2770 & 18,5 & 975  & 6,5 & 1790 \\
        \hline
        \underline{$\text{Gd}_{2}\text{O}_{3}$} & 11,5 & 2575 & 13,5 & 1825 & 2,0 & 750 \\
                                                & 12,5 & 2650 & 13,5 & 2070 & 1,0 & 580 \\
                                                & 12,5 & 2645 & 13,0 & 1850 & 0,5 & 795 \\
        \hline
        \underline{$\text{Nd}_{2}\text{O}_{3}$} & 11,5 & 2650 & 12,5 & 2385 & 1,0  & 265 \\
                                                & 12,5 & 2715 & 12,0 & 2420 & -0,5 & 295 \\
                                                & 12,0 & 2610 & 12,0 & 2385 & 0    & 225 \\
        \bottomrule
    \end{tabular} 
\end{table}

\begin{flushleft}
    Die Werte werden in Tabelle \ref{Tabelle4} und \ref{Tabelle5} gemittelt und die Standardabweichung gebildet.
\end{flushleft}


\begin{table}[H]
    \centering
    \caption{Mittelwerte und Abweichungen der Spannungen.} 
    \label{Tabelle4}
    \begin{tabular} {c || c  c  c }
        \toprule
        {Ionen} &
        {$ \text{U}_{\text{oP}} \mathbin{/} 10^{-3}\,\unit{\volt} $} &
        {$ \text{U}_{\text{mP}} \mathbin{/} 10^{-3}\,\unit{\volt} $} &
        {$ \increment\text{U} \mathbin{/} 10^{-3}\,\unit{\volt} $} \\
        \midrule
        $\text{Dy}_{2}\text{O}_{3}$ & $12,50 \pm 0,40$ & $17,16 \pm 0,24$ & $4,60 \pm 1,43$ \\
        $\text{Gd}_{2}\text{O}_{3}$ & $12,16 \pm 0,47$ & $13,30 \pm 0,23$ & $1,16 \pm 0,62$ \\
        $\text{Nd}_{2}\text{O}_{3}$ & $12,00 \pm 0,40$ & $12,16 \pm 0,23$ & $0,16 \pm 0,62$ \\
        \bottomrule
    \end{tabular} 
\end{table}

\begin{table}[H]
    \centering
    \caption{Mittelwerte und Abweichungen der Widerstände.} 
    \label{Tabelle5}
    \begin{tabular} {c || c  c  c }
        \toprule
        {Ionen} &
        {$ \text{R}_{\text{oP}} \mathbin{/} 10^{-3}\,\unit{\ohm} $} &
        {$ \text{R}_{\text{mP}} \mathbin{/} 10^{-3}\,\unit{\ohm} $} &
        {$ \increment \text{R}  \mathbin{/} 10^{-3}\,\unit{\ohm} $}  \\
        \midrule
        $\text{Dy}_{2}\text{O}_{3}$ & $2555 \pm 166,74$ & $1005   \pm 32,40$  & $1550  \pm 194,2$ \\
        $\text{Gd}_{2}\text{O}_{3}$ & $2623 \pm 34,23$  & $1915   \pm 110,07$ & $708,3 \pm 92,58$ \\
        $\text{Nd}_{2}\text{O}_{3}$ & $2658 \pm 43,26$  & $2396,6 \pm 272,37$ & $261,6 \pm 28,67$ \\ 
        \bottomrule
    \end{tabular} 
\end{table}

\begin{table}[H]
    \centering
    \caption{Suszeptibilität der seltenen Erd-Elemente.} 
    \label{Tabelle6}
    \begin{tabular} {c |  c  c  c}
        \toprule
        {Ionen} &
        {$ \chi_{\text{U}} $} &
        {$ \chi_{\text{R}} $} \\
        \midrule
        $\text{Dy}_{2}\text{O}_{3}$ & $0,012900 \pm 0,0030590$ & $0,02135 \pm 0,003180$ \\
        $\text{Gd}_{2}\text{O}_{3}$ & $0,004770 \pm 0,0032715$ & $0,01013 \pm 0,001571$ \\
        $\text{Nd}_{2}\text{O}_{3}$ & $0,000352 \pm 0,0018523$ & $0,00250 \pm 0,000347$ \\
        \bottomrule
    \end{tabular} 
\end{table}

\subsection{Theoretische Berechnung der Suszeptibilität}

\begin{align*}
    \intertext{Die Suszeptibilität berechnet sich theoretisch nach der Formel (\ref{9}).
    Aus den Hund'schen Regeln ergeben sich dann die Werte für Spin, Drehimpuls, Gesamtdrehimpuls und der daraus resultierende Landé-Faktor, welcher sich nach Formel (\ref{7}) berechnen lässt.
    Dabei enthält $\text{Nd}^{3+}$ drei, $\text{Gd}^{2+}$ sieben und die $\text{Dy}^{3+}$ -Hülle neun 4f-Elektronen. 
    Die Daten werden in Tabelle \ref{Tabelle7} aufgenommen.
    Als Temperatur wird der Wert $\text{T} = 293\,\unit{\kelvin}$ verwendet.
    N ist die Anzahl der Momente pro Volumeneinheit und berechnet sich nach}
    \text{N} = \frac{\delta_{\text{w}}}{\text{M}} \,.
    \intertext{ M ist dabei die Molare Masse.
    Daraus folgt}
    \text{N}_{\text{Dy}_{2}\text{O}_{3}} = 2,52 \cdot 10^{28}\,\frac{1}{\unit{\meter}^3} \\
    \text{N}_{\text{Gd}_{2}\text{O}_{3}} = 2,46 \cdot 10^{28}\,\frac{1}{\unit{\meter}^3} \\ 
    \text{N}_{\text{Nd}_{2}\text{O}_{3}} = 2,59 \cdot 10^{28}\,\frac{1}{\unit{\meter}^3}\,. \\
\end{align*}

\begin{table}[H]
    \centering
    \caption{Quanteneigenschaften.} 
    \label{Tabelle7}
    \begin{tabular} {c|  c  c  c  c}
        \toprule
        { } &
        {$\text{Dy}_{2}\text{O}_{3}$} &
        {$\text{Gd}_{2}\text{O}_{3}$} &
        {$\text{Nd}_{2}\text{O}_{3}$} \\
        \midrule
        4f-Elektron                        & 9,0  & 7,0 & 3,0  \\
        Spin S                             & 2,5  & 5,0 & 3,5  \\
        Drehimpuls L                       & 5,0  & 0   & 6,0  \\
        Gesamtdrehimpuls J                 & 7,5  & 3,5 & 4,5  \\
        Landé-Faktor $\text{g}_{\text{i}}$ & 1,33 & 2,0 & 0,73 \\
        \bottomrule
    \end{tabular} 
\end{table}

\begin{align*}
    \intertext{Aus den Quanteneigenschaften folgt für die theoretische Werte der Suszeptibilität:}
    \chi_{\text{T},\text{Dy}_{2}\text{O}_{3}} = 0,0254 \\
    \chi_{\text{T},\text{Gd}_{2}\text{O}_{3}} = 0,0138 \\
    \chi_{\text{T},\text{Nd}_{2}\text{O}_{3}} = 0,0030\,. 
\end{align*}
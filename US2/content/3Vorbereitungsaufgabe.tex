\section{Vorbereitungsaufgabe}

\begin{align}
    \intertext{Bei der ersten Vorbereitungsaufgabe sollen die Literaturwerte der Schallgeschwindigkeit zu Luft, destiliertem Wasser und Acryl recherchiert werden.
    Diese werden im folgenden Aufgezählt. }
    \text{c}_{\text{Luft}} = 344\, \frac{\unit{\meter}}{\unit{\second}}\,\,\text{\cite{a3}} \notag\\
    \text{c}_{\text{Wasser}} = 1485\, \frac{\unit{\meter}}{\unit{\second}}\,\,\text{\cite{a3}} \notag\\
    \text{c}_{\text{Acryl}} = 2730\, \frac{\unit{\meter}}{\unit{\second}}\,\,\text{\cite{a2}} \notag\\
    \intertext{Bei der zweiten Vorbereitungsaufgabe sollen die Wellenlängen und die Periode für $1\,\unit{\mega\hertz}$, $2\,\unit{\mega\hertz}$ und $4\,\unit{\mega\hertz}$ Schwingungen in Acryl berechnet werden.
    Dafür benötigt man die beiden Gleichungen (\ref{7}) und (\ref{8}) }
    \text{T} = \frac{1}{\text{T}} \label{7}\\ 
    \lambda = \frac{\text{c}}{\text{f}} \label{8}\\ \notag
\end{align}

\begin{table}[H]     
    \centering
    \caption{Die einzelnen Werte zu der zweiten Vorbereitungsaufgabe } 
    \label{Tabelle1}
    \begin{tabular} {c ||  c  c  c}
        \toprule
        {$  $} &
        {$ 1\,\unit{\mega\hertz} $} &
        {$ 2\,\unit{\mega\hertz} $} &
        {$ 4\,\unit{\mega\hertz} $} \\
        \midrule
        f         & $1 \cdot 10^{6}$ & $2 \cdot 10^{6}$ & $3 \cdot 10^{6}$ \\
        T         & $\frac{1}{1 \cdot 10^{6}}$ & $\frac{1}{2 \cdot 10^{6}}$ & $\frac{1}{4 \cdot 10^{6}}$ \\
        \\
        $\lambda$ & $2,73 \cdot 10^{-3}$ & $1,365 \cdot 10^{-3}$ & $6,825 \cdot 10^{-5}$ \\
    \end{tabular} 
\end{table}

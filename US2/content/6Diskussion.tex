\section{Diskussion}

\begin{align*}
    \intertext{Die Bestimmung der Schallgeschwindigkeit weist auf eine relative Abweichung von}
    \text{c}_{\text{lit}} = 2730\,\frac{\unit{\meter}}{\unit{\second}}  \,\,\,\,\,\, \text{c}_{\text{exp}} = (2961,74 \pm 121,36)\,\frac{\unit{\meter}}{\unit{\second}}\\
    \to 8,48\%
    \intertext{Trotz der Ungenauigkeit ist der Wert akzeptabel, dennoch werden für die weiteren Aufgaben der Literatur angenommen.}
\end{align*}

\begin{flushleft}
    Auffällig bei der Versuchsreihe war, dass die Bestimmung der Fehlstellen und Laufzeiten problemlos liefen.
    Dies führt dazu, dass gewisse Abweichungen wie in Tabelle \ref{Tabelle6} auftreten.
\end{flushleft}

\begin{table}[H]
    \centering
    \caption{Abweichung der Messwerte.} 
    \label{Tabelle6}
    \begin{tabular} {c  c  c  c  c  c}
        \multicolumn{3}{|c|}{Tabelle \ref{Tabelle3} verglichen mit den Literaturwerten \cite{a2} in \%} &
        \multicolumn{2}{|c|}{Tabelle \ref{Tabelle5} verglichen mit Tabelle \ref{Tabelle3} in \%} \\
        \hline
        {$ \text{Loch} $} &
        {$ \text{s}_{\text{oben}} \mathbin{/} \unit{\milli\meter} $} &
        {$ \text{s}_{\text{unten}} \mathbin{/} \unit{\milli\meter} $} &
        {$ \text{s}_{\text{oben}} \mathbin{/} \unit{\milli\meter} $} &
        {$ \text{s}_{\text{unten}} \mathbin{/} \unit{\milli\meter} $} \\
        \midrule
        4  & 2,55   & 5,45 & 2,49 & 4,69 \\
        5  & 2,51   & 3,73 & 8,97 & 3,05 \\
        6  & 4,37   & 3,83 & 3,76 & 3,10 \\
        7  & 5,90   & 3,22 & 3,81 & 3,17 \\
        8  & 9,06   & 2,56 & 5,04 & 3,43 \\
        9  & 14,52  & 0,95 & 8,37 & 5,37 \\
        10 & 102,03 &   /  & 29,4 &   /  \\
        \bottomrule
    \end{tabular} 
\end{table}

\begin{flushleft}
    Die Abweichungen, wie der A-Scan mit den Literaturwerten sowohl als auch der B-Scan mit den A-Scan, sind sehr gering.
    Dadurch sind die Werte in einem annehmbaren Bereich und legitim.
    Bei Lochnummer zehn sticht der Wert mit 102\% heraus, welche auf eine mögliche Fehlerquelle deutet.
    Angenommen waren Loch drei bis zehn, wobei die zehn bei gedrehtem Block nicht messbar war. 
    Zudem kommen gewisse Ablesefehler am Monitor zustande, wie das nicht präzise setzen des Cursors auf die Fehlstelle.
    Dabei war die Grafik ständig am Arbeiten.
    Dennoch sind die Werte akzeptabel.
    Im zweiten Teil der Aufgabe ist auffällig, dass die Grafik mit der $ 1\,\unit{\mega\hertz} $ Sonde deutlich breiter angeordnet ist, wodurch die Messung der Laufzeit und Störstelle schwerer fiel.
    Bei der $ 2\,\unit{\mega\hertz} $ Sonde waren die Peaks deutlicher zu erkennen, was dazu führt, dass diese Messung mit einer $ 2\,\unit{\mega\hertz} $ Sonde deutlich effektiver ist.
    Grundsätzlich eignen sich beide Methoden, A-Scan und B-Scan, zur Untersuchung der Störstellen aber dennoch können kleine systematische Fehler auftreten.
\end{flushleft}

\begin{flushleft}
    Im letzten Teil der Aufgabenstellung war die Untersuchung eines Tumors an einem Brustmodell. 
    Jedoch waren die Methoden mit der vorhandenen Apparatur nicht funktionstüchtig. 
    Das Ertasten des Tumors war aufklärend. Die Ermittlung des Tumors wurde mit Hilfe von mehreren Scans durchgeführt andererseits ohne einen klare Feststellung. 
    Dementsprechend wurde die letzte Aufgabe nur ausprobiert und in der Auswertung sowie Durchführung nicht erwähnt.
\end{flushleft}
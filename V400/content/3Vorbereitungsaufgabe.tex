\section{Vorbereitungsaufgabe}

\begin{flushleft}
    Bei der ersten Vorbereitungsaufgabe sollen die Literaturwerte für die Brechungsindizes, in der Tabelle \ref{Tabelle1} gelisteten, Materialien recherchiert werden. \cite{a2}
\end{flushleft}

\begin{table}[H]
    \centering
    \caption{Brechungsindizes der Materialien} 
    \label{Tabelle1}
    \begin{tabular} {c | c}
        \toprule
        {Material} &
        {Brechungsindex n} \\
        \midrule
        Luft      & 1,00027461 \\
        Wasser    & 1,3150     \\
        Kronglas  & 1,52       \\
        Plexiglas & 1,48999    \\
        Diamant   & 2,3866     \\
        \bottomrule
    \end{tabular} 
\end{table}

\begin{align*}
    \intertext{Bei der zweiten Vorbereitungsaufgabe sollen die Gitterkonstante d für die unten aufgelisteten Gitter berechnet werden}
    600\, \text{Linien} \mathbin{/} \unit{\milli\meter} \implies \text{d} = \frac{1}{600}\,\unit{\micro\meter}\,, \\
    300\, \text{Linien} \mathbin{/} \unit{\milli\meter} \implies \text{d} = \frac{1}{300}\,\unit{\micro\meter}\,, \\
    100\, \text{Linien} \mathbin{/} \unit{\milli\meter} \implies \text{d} = \frac{1}{100}\,\unit{\micro\meter}\,.
\end{align*}
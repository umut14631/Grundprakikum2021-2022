\section{Diskussion}

\begin{align*}
    \intertext{Durch die Versuchsreihe wurden die grundlegenden Gesetzmäßigkeiten der Strahlenoptik verdeutlicht.
    Beim Reflexionsgesetz gilt die Voraussetzung $\alpha_{1} = \alpha_{2}$. 
    Die Messdaten bewiesen diese Bedingung, jedoch ergaben sich zwei Ungenauigkeiten, die wiederum auf einen mögliche Ablesefehler hindeuten.}
    \intertext{Die relative Abweichung berechnet sich über}
    \frac{\text{x}_{\text{exp}} - \text{x}_{\text{theo}}}{\text{x}_{\text{theo}}} \cdot 100 = \text{prozentuale Abweichung}\,. \label{10}
    \intertext{Der durch das Brechungsgesetz hergeleitete Brechungsindex liefert die folgende Abweichung}
    \text{n}_{\text{exp}} = (1,4871 \pm 0,0116)\,\,\,\,\,\,\,  \text{n}_{\text{Lit, Plex.}} = 1,48899\,\,\,\, \text{\cite{a2}} \notag \\
    \to 32,75\% \notag 
    \intertext{Die Abweichung ist sehr gering, somit stellt sich die Methode als erfolgreich dar.
    Durch die Brechung sollte die Anfangsgeschwindigkeit von $\text{c} = 2,9979 \cdot 10^{8}\frac{\unit{\meter}}{\unit{\second}}$ größer sein als die Geschwindigkeit nach der Brechung.
    Die Voraussetzung $\text{c}_{1} > \text{c}_{2}$ wurde erfüllt mit der Abweichung}
    \text{c}_{2} =  2,0159 \cdot 10^{8}\frac{\unit{\meter}}{\unit{\second}}\,\,\,\,\,\,\, \text{c}_{1} = 2,9979 \cdot 10^{8}\frac{\unit{\meter}}{\unit{\second}} \\
    \to 32,75\%
    \intertext{Wie erwartet verkleinert sich der Strahlversatz bei jeder Verkleinerung des Einfallswinkels.
    Die Berechnung am Prisma zeigte, dass sie bei fester Wellenlänge von Einfallswinkeln $\alpha_{1}$ vom brechenden Winkel $\gamma$ und von der Brechzahl n des Prismas abhängt.
    Der Unterschied ist, dass der Ablenkwinkel beim roten Later minimal kleiner ist als beim grünen Laser.}
    \intertext{Durch das Beugungsgitter konnten folgende Wellenlängen mit der Abweichung untersucht werden}
\end{align*}

\begin{table}[H]
    \centering
    \caption{Abweichung der Wellenlängen vom roten Laser} 
    \label{Tabelle6}
    \begin{tabular} {c | c}
        \toprule
        {$ $} &
        {$ \lambda_\text{Lit, rot} = 638\,\unit{\nano\meter} \text{\cite{a1}} \,\,\,\, \text{Abweichung in \%} $} \\
        \midrule
        $\alpha = 656\,\unit{\nano\meter}$ & 2,8 \\
        $\alpha = 621\,\unit{\nano\meter}$ & 2,6 \\
        $\alpha = 630\,\unit{\nano\meter}$ & 1,2 \\
    \end{tabular} 
\end{table}

\begin{table}[H]
    \centering
    \caption{Abweichung der Wellenlängen vom grünen Laser} 
    \label{Tabelle7}
    \begin{tabular} {c | c}
        \toprule
        {$ $} &
        {$ \lambda_\text{Lit, grün} = 532\,\unit{\nano\meter} \text{\cite{a1}} \,\,\,\, \text{Abweichung in \%} $} \\
        \midrule
        $\alpha = 556\,\unit{\nano\meter}$ & 4,5 \\
        $\alpha = 519\,\unit{\nano\meter}$ & 2,4 \\
        $\alpha = 519\,\unit{\nano\meter}$ & 2,4 \\
    \end{tabular} 
\end{table}

\begin{flushleft}
    Die Abweichungen sind minimal zum Literaturwert.
    Dennoch fiel das Ablesen von dem Skala schwer, dar die Vorlage richtig positioniert werden musste.
    Die Apparatur durfte keine Bewegung erfassen, da sonst sich die Skala verschoben hätte.
    Nichtsdestotrotz sind die Abweichungen bei allen Versuchen gering und somit erwiesen sich die Methoden als profitabel für das Kennenlernen der Gesetzmäßigkeiten der Strahlenoptik.
\end{flushleft}
\section{Diskussion}

\begin{flushleft}
    Anhand der drei Versuchsteile konnten drei verschiedene Werte für die Zeitkonstante ermittelt werden.
    Beim Vergleich der verschiedenen Zeitkonstanten folgt für die relativen Abweichungen:
\end{flushleft}

\begin{table}[H]
    \centering
    \label{Tabelle3}
    \begin{tabular} {c||  c|  c|  c}
        \toprule
        {$  $} &
        {$ RC_{1} $} &
        {$ RC_{2} $} &
        {$ RC_{3} $} \\
        \midrule
        $RC_{1}$ &   -   & 88\% & 45\%  \\
        $RC_{2}$ & 707\% &  -   & 345\% \\
        $RC_{3}$ & 81\%  & 78\% &  - \\
        \bottomrule
    \end{tabular} 
\end{table}

\begin{flushleft}
    Auffällig hierbei ist, dass die Zeitkonstanten stark voneinander abweichen, die zum einen auf die Ablesefehler der einzelnen Werte zurückzuführen sein können.
    Das Oszilloskop hat ebenso einen Effekt auf die Messwerte, denn bei $20\,\unit{\kilo\hertz}$ schwankte der Graph, wodurch das Ablesen schwer fiel.
\end{flushleft}

\begin{flushleft}
    Bei genauer Betrachtung weichen einige Messwerte von der Theorie ab, wie bei der linearen Regression oder der Phasenverschiebung, die teilweise durch den Ablesefehler erklärt werden können.
    Wie zu erwarten, sieht die Kondensatorspannung mit zunehmende Frequenz trotzdem ab und die Phasenverschiebung bleibt im Bereich null bis $\frac{\pi}{2}$.
    Im Vergleich ist es zu erkennen, dass $RC_{2}$ die kleinste Abweichung zu den anderen besitzt und somit der Wert am ehesten angenommen werden kann.
    Bezüglich des Integratorverhaltens, lässt sich sagen, dass der Erwartungswert mit dem Realwert des Oszilloskops übereinstimmt und somit die Integrierbarkeit des $RC$-Kreises bewiesen wurde.
\end{flushleft}
\section{Vorbereitungsaufgabe}

\begin{flushleft}
    Als erste Vorbereitungsaufgabe für diesen Versuch sollen die Energien für Kupfer $K_{\alpha}$ und $K_{\beta}$ Linien recherchiert werden, und dann die dazugehörige Wellenlänge und den Winkel $\alpha$ für eine LiF-Kristall $(d_{\text{LiF}} = 201.4\,\unit{\pico\meter})$ mit $\text{n} = 1$ bestimmt werden.
    Danach sollen die dazugehörigen Compton-Wellenlängen berechnet werden.
\end{flushleft}

\begin{align*}
    \intertext{Die Energien der $K_{\alpha}$ und $K_{\beta}$ Linien sind: \cite{a2}}
    E_{K_{\alpha}} = 8010\,\text{eV} \\
    E_{K_{\beta}} = 8862\,\text{eV}
    \intertext{durch die Formel für die Photonenenergie}
    E = \frac{h \cdot c}{\lambda}
    \intertext{lässt sich, nach Umstellung nach $\lambda$, die zugehörige Wellenlänge berechnen}
    \lambda = \frac{h \cdot c}{E} \\
    \lambda_{\alpha} = 154.78\,\unit{\pico\meter} \\
    \lambda_{\beta} = 139.90\,\unit{\pico\meter}\,. \\
    \intertext{Durch die Gleichung (\ref{3}) folgt der Winkel \alpha}
    \alpha_{\alpha} = 23.12 \unit{\degree}\\
    \alpha_{\beta} = 20.98\unit{\degree}\,. \\
    \intertext{Die Compton-Wellenlänge wird mit}
    \lambda_{\text{c}} = \frac{h}{m_{\text{e}}c} 
    \intertext{berechnet und ergibt}
    \lambda_{\text{c}} = (2.4263 \cdot 10^{-12})\,\unit{\meter}\,.
\end{align*}
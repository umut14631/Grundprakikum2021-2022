\section{Diskussion}

\begin{align*} 
    \intertext{Beim Vergleich der folgenden Energien der $K_{\alpha}$ und $K_{\beta}$ Linien mit den jeweiligen Literaturwerten \cite{a2}, stellt sich eine relative Abweichung von}
    E_{K_{\alpha},\text{theo}} = 8010\,\text{eV} \hspace{1cm} E_{K_{\alpha},\text{exp}} = (7945.06 \pm 19.43)\,\text{eV} \\
    \\
    \quad \to \text{Abweichung = 0.8\% }\\
    \\
    E_{K_{\beta},\text{theo}} = 8862\,\text{eV} \hspace{1cm} E_{K_{\beta},\text{exp}} = (8805.26 \pm 24.24)\,\text{eV}\\
    \\
    \quad \to \text{Abweichung = 0.6\%}\\
    \intertext{dar. Die relative Abweichung ist sehr niedrig, was dazu führt, dass die experimentell bestimmten Energien sehr nah an den Literaturwerten liegen.
    Das Ergebnis war zu erwarten, da 170 Messpaare aufgenommen werden, die das Ergebnis genauer eingrenzen.
    Durch die sehr kleine Abweichung befinden sich die Ergebnisse im Toleranzbereich und der Versuch ist somit gut einsetzbar für die Bestimmung der Energien.}
\end{align*}

\begin{align*}
    \intertext{Die experimentell bestimmte Comptonwellenlänge verglichen mit dem Literaturwert \cite{a3}, liefert folgende Abweichung:}
    \lambda_{exp}= (3.77 \pm 0.082)\,\unit{\pico\meter} \,\,\,\,\,\,\,\, \lambda_{theo}= (2.43)\,\unit{\pico\meter}  \\
    \\
    \quad \to \text{Abweichung = 55.14\%\,.}
    \intertext{Die Abweichung beträgt knapp über die Hälfte und befindet sich nicht im Fehlerbereich.
    Mögliche Fehlerquellen wäre eine falsche Einstellung der Messsgeräte oder die falsche Kalibrierung der Röntgenröhre.
    Es ist zu beachten, dass der Versuch nicht selbstständig durchgeführt wurde und der Verdacht eher auf den Versuchsaufbau für den Fehler liegt.}
\end{align*}

\begin{flushleft}
    Auf die Frage, warum der Compton-Effekt nicht im sichtbaren Bereich des Spektrums auftreten kann, lässt sich sagen, dass die Skala für den Menschen sichtbare Bereich höher ist $(400\,\unit{\nano\meter}\,-\,700\,\unit{\nano\meter})$.
    Die maximale Wellenlängenverschiebung erfolgt bei $\theta = 180\unit{\degree}$, die sich durch die Formel $\increment \lambda = 2 \cdot \lambda_{c}$ berechnen lässt.
    Trotz der maximalen Größe von $\increment \lambda = 2 \cdot \lambda_{c} = (4,86)\,\unit{\pico\meter}$ ist sie nicht im sichtbaren Bereich.
    Die Compton-Wellenlänge ist somit mehrere Größenordnungen zu klein. 
\end{flushleft}
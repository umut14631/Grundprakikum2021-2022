\section{Diskussion}

\begin{flushleft}
    Folgende Werte und Abweichungen von den Literaturwerten wurden bestimmt
\end{flushleft}

\begin{table}[H]
    \centering
    \caption{Abweichung in \%} 
    \label{Tabelle}
    \begin{tabular} {c c c}
        \toprule
        { } &
        {$ \mu_\text{com,Eisen} $} &
        {$ \mu_\text{com,Blei} $} \\
        \midrule
        $\mu_{\text{Eisen}}$ & 26,7 & - \\
        $\mu_{\text{Blei}}$ & - & 32,3 \\
        \bottomrule
    \end{tabular} 
\end{table}

\begin{flushleft}
    Obwohl die Absorptionskoeffizienten auf dem ersten Blick positiv erscheinen, weisen sie dennoch gewisse Ungenauigkeiten auf, die gerade nicht gering sind.
    Die Ungenauigkeiten könnten auf die Apparatur zurückgeführt werden, wie z.B. bei der Messung der Zählrate. 
    Die Materialien üben ebenso einen Effekt auf die Messung aus. 
    Die Oberfläche von Blei war brüchig oder es ist anzunehmen, dass Blei einen deutlich schwereren Kern besitzt und somit der Photoeffekt einen größeren Effekt hat.
\end{flushleft}

\begin{flushleft}
    Auffällig bei der Messung der $\beta$-Strahlung war, dass die Zählrate mit dem Timer ab der Hälfte der Messung nicht übereinstimmte. 
    Mögliche Fehlerquelle könnte die instabile Natur des Zerfalls gewesen sein. 
    Um diese entgegenzukommen wurde die Dauer immer um $50\,\unit{\second}$ erweitert.
\end{flushleft}

\begin{flushleft}
    Nichtsdestotrotz eignen sich die Methoden zur Untersuchung des Absorptionsgesetzes und der Maximalenergie eines $\beta$-Strahlers.
    Es ist jedoch anzumerken, dass die menschlichen Hände nicht präzise sind und somit Ungenauigkeiten auftreten können.
\end{flushleft}
\section{Auswertung} 

\begin{align}
    \intertext{Zur Bestimmung der Wellenlänge des Lasers werden die Interferenzmaxima z, sowie die dazugehörige Verschiebung des Spiegels d benötigt.
    Aus der Verschiebung d und der Voreinstellung $ \text{d}_{\text{v}} = 7 \cdot 10^{-3}\,\unit{\meter}$ resultiert die Differenz $\increment \text{d}$ der Verschiebung des Spiegel.
    Beachtet wird dabei die Hebelübersetzung von $\overline{\text{U}} = 5,017$. 
    Die Verschiebung wird um den Faktor $\frac{1}{\overline{\text{U}}}$ erweitert. 
    Durch das Umformen der Formel (\ref{5}) nach $\lambda$ wird aus den Werte paaren die Wellenlänge gewonnen. 
    Die entsprechende Messwerte und Wellenlängen folgen in Tabelle \ref{Tabelle1}. 
    Die Wellenlängen werden gemittelt und die dazugehörige Abweichung wird bestimmt, nach}
    \overline{\text{x}} = \frac{1}{\text{N}} \sum_{\text{i} = 1}^{\text{N}} \text{x}_{\text{i}}\,, \label{7} \\
    \increment \overline{\text{x}} = \frac{1}{\sqrt{\text{N}}} \sqrt{ \frac{1}{\text{N}-1} \sum_{\text{i} = 1}^{\text{N}} \left(\text{x}_{\text{i}} - \overline{\text{x}}\right)^{2}}\,. \label{8}
\end{align}

\begin{table}[H]
    \centering
    \caption{Die Messwerte und die Wellenlängen.} 
    \label{Tabelle1}
    \begin{tabular} {c  c   c}
        \toprule
        {$ \text{d} \mathbin{/} 10^{-3}\,\unit{\meter} $} &
        {$ \text{z} $} &
        {$ \lambda \mathbin{/} 10^{-9}\,\unit{\meter} $} \\
        \midrule
        3,65 & 2002 & 666,33 \\
        3,65 & 2000 & 667,00 \\
        3,65 & 2001 & 666,66 \\
        3,66 & 2001 & 664,66 \\
        3,70 & 2006 & 655,03 \\
        3,66 & 2004 & 663,67 \\
        3,53 & 2001 & 691,30 \\
        3,58 & 2001 & 681,34 \\
        \bottomrule
    \end{tabular} 
\end{table}

\begin{align}
    \intertext{Daraus folgt für die Wellenlänge}
    \lambda = (669,49 \pm 10,64) \cdot 10^{-9}\,\unit{\meter}\,.
    \intertext{Um den Brechungsindex von Luft zu bestimmen werden zunächst folgende Größen benötigt:}
    \text{Normaldruck} \,\,\, \text{p}_{0} = 1,0132\,\unit{\bar} \notag \\
    \text{Normaltemperatur} \,\,\, \text{T}_{0} = 273,15\,\unit{\kelvin}\notag  \\
    \text{Umgebungstemperatur} \,\,\, \text{T} = 293,15\,\unit{\kelvin}\notag  \\
    \text{Größe der Messzelle} \,\,\, \text{b} = 50 \cdot 10^{-3}\,\unit{\meter}\,.\notag
    \intertext{Bei der Messung wird der Kammerdruck p und die Inferenzmaxima z beim Einlassen von Luft beachtet.
    Die Berechnung des Brechungsindex von Luft erfolgt nach der Formel (\ref{10}).
    Die Messdaten und die dazugehörigen Brechungsindizes n befinden sich in Tabelle \ref{Tabelle2}. } \notag
\end{align}

\begin{table}[H]
    \centering
    \caption{Die Messwerte zur Berechnung des Brechungsindex von Luft.} 
    \label{Tabelle2}
    \begin{tabular} {c  c  c  c}
        \toprule
        {$ \text{p} \mathbin{/} \unit{\bar} $} &
        {$ \text{z} $} &
        {$ \text{n} $} \\
        \midrule
        0,6 & 15 & 1,00055 \\
        0,6 & 16 & 1,00058 \\
        0,6 & 16 & 1,00058 \\
        0,6 & 16 & 1,00058 \\
        0,6 & 16 & 1,00058 \\
        \bottomrule
    \end{tabular} 
\end{table}

\begin{align*}
    \intertext{Nach Gleichung (\ref{7}) und (\ref{8}) folgt für den Brechungsindex von Luft}
    \text{n} = (1,000574 \pm 0,000014)\,.
\end{align*}
\section{Diskussion}

\begin{align}
    \intertext{Das Justieren der Apparatur erfolgte problemlos.
    Dennoch sollte beachtet werden, dass das Interferenzmuster bzw. der Laser, die Sonde trifft, sonst werden keine Interferenzmuster gezählt.
    Die relative Abweichung berechnet sich über}
    \frac{\text{x}_{\text{exp}} - \text{x}_{\text{theo}}}{\text{x}_{\text{theo}}} \cdot 100 = \text{prozentuale Abweichung}\,.
    \intertext{Die Berechnung des Brechungsindex von Luft sorgte für folgende Abweichung:}
    \text{n}_{\text{exp}} = (1,000574 \pm 0,000014) \,\,\,\,\,\,\, \text{n}_{\text{theo}} = 1,00027611\,\, \text{\cite{a2}} \notag \\
    \to 0,029\% \notag
    \intertext{Die Bestimmung der Wellenlänge führt zur folgender Abweichung }
    \lambda = (669,49 \pm 10,64) \cdot 10^{-9}\,\unit{\meter} \,\,\,\,\,\,\, \lambda_{\text{Lit}} = 635\,\unit{\nano\meter}\notag\\
    \to 5,43\%\notag
    \intertext{Die Ungenauigkeit kann auf den berechneten Wert der Wellenlänge führen, wie z.B die Ungenauigkeit beim Ablesen der Verschiebung des Spiegels.
    Beim Aufnehmen der Interferenzmaxima ging die Zählung trotz der 2000er Grenze weiter. 
    Das führt dazu, dass eine gewisse Anzahl an Impulsen zur Abweichung führen könnten.
    Das Füllen der Kammer mit Luft könnte ebenso zur Abweichungen führen aber dennoch sind die Wertepaare in einem akzeptablen Bereich.}\notag
\end{align}
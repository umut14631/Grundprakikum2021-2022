\section{Vorbereitungsaufgaben}
\begin{flushleft}
    Als Vorbereitungsaufgabe für diesen Versuch gibt es zwei Fragen zu lösen.
    Die Fragen sind, wann von einer harmonischen Schwingung gesprochen wird und für welche Winkel die kleinwinkelnäherung noch gilt 
    bei einem Fadenpendel der länge $ 70 \, \unit{\centi\meter} $. 
    Eine gekoppelte Schwingung ist definiert als, Schwingung wobei die rücktreibende Kraft proportional zur auslenkung ist. 
    Die Winkel für welche die kleinwinkelnäherung gilt, sind nicht wirklich festgelegt, solange der Winkel klein gehalten wird. 
    Dies geht jedoch bis zu $ 10 \unit{\degree}$ Auslenkung aus. 
\end{flushleft}
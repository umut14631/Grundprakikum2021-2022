\section{Diskussion}

\begin{flushleft}
    Bei kronologischer Vorgehensweise, wird bei der Bestimmung der Regression gesehen, dass die Werte in einem annehmbaren Bereich liegen.
    Wie zu erkennen verläuft die Gerade recht \enquote{gut} durch die Punkte. 
    Ebenso festzulegen durch die Kennzahl beziehungsweise das Bestimmtheitmaß $ r^2 = 0,987 $, welche genau darstellt, wie die Werte zum Modell passen. 
    Vergleich bei $ r^2 = 1 $ würde es heißen, dass die Gerade durch alle Punkte gehen würden. Durch die Abweichung lassen sich sogenannte Quadrate darstellen, in unserem Fall: $ \sum \limits Quadrate = 40,4507 $.
    Durch das Vergleichen der Werte der Objekte mit den theoretisch sowie experimentell bestimmten werten und der verschiedenen Körperhaltung fällt auf, dass der Zylinder eine relative Abweichung, berechnet durch die Formel (\ref{8}), von $ -37,76 \, \% $ hat. 
    Für die Kugel beträgt die Abweichung des experimentellen Wertes von den Theoretischen um $ -21,24 \, \% $. 
    Die erste Körperhaltung der Puppe hat eine Abweichung von $ -8,35 \, \% $ und die zweite eine von $ -17,8 \,\% $. 
    Bei der Untersuchung der Äquivalenz der Trägheitsmomente der zwei Haltungen führt dazu, dass die erste Haltung kleiner ist, um $ \frac{1}{2} $, als die zweite Haltung. 
    Beim genaueren Betrachten fällt auf, dass die relative Abweichung im negativen Bereich befindet. 
    Dies weißt daraufhin, dass die theoretischen Werte größer sind als die experimentell bestimmten Werte. 
    Wie schon in der Durchführung thematisiert, handelt es sich um einen Versuch, wobei viel abgelesen werden muss. 
    Dadurch entstehen Fehler beim Ablesen der Schwingungsdauer, des Kraftmessers oder bei der Schieblehre.
    Zudem die genaue zeitliche Koordinierung beim Auslenken und stoppen der Uhrzeit stellt eine große Fehlerquelle dar. 
    Somit sind systematische Fehler vorhanden.
\end{flushleft}
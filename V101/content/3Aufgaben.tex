\section{Aufgaben}

\begin{flushleft}
    Zunächst wird die Winkelgröße $ D $ und das dazugehörige Eigenträgheitsmoment ${l}_{\text{D}}$ durch Anwendung des Satzes von Steiner, bestimmt.
    Danach soll das Trägheitsmoment $ l $ von den zwei Körpern bestimmen und mit den theoretisch berechneten Werten vergleichen. 
    Als letztes soll das Trägheitsmoment einer Modellpuppe bestimmt werden, welche in zwei verschiedene Haltungen eingestellt wird.
    Die ermittelten Werte sollen dann verglichen werden mit der dazugehörigen Modellrechnung. 
\end{flushleft}
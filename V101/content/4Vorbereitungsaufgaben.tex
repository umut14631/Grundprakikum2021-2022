\section{Vorbereitungsaufgaben}

\begin{align}
    \intertext{Die Vorbereitungsaufgabe für diesen Versuch besteht aus dem Berechnen des Drehmoments $ M $.
    An einer Stange wirkt Senkrecht eine Kraft von $ F = 0.1 \,\unit{\newton}$, in einem Winkel von $ \alpha_{1} = 90 \unit{\degree} $. 
    Ebenso soll man $ \alpha_{2} = 45 \unit{\degree} $
    Die Werte sollen in einem Abstandsbereich von 5 bis 25\,cm bestimmt werden.
    Dies soll für 10 verschiedene Abstände berechnet werden. Hierfür wird die Formel}
    M = F \cdot r \cdot \sin{\alpha} \notag
    \intertext{benötigt.}
    \to \quad M_{r1} = 0.1\unit{\newton} \cdot r \cdot \sin{(45 \unit{\degree})} \\
    \to \quad M_{r2} = 0.1\unit{\newton} \cdot r \cdot \sin{(90 \unit{\degree})} 
\end{align}

\begin{align*}
\intertext{die verschiedenen Abstände eingesetzt führen zu den Ergebnissen }
    M_{r1=5}  = 0,353\, \unit{\newton}, \, \, \, \, M_{r2=5}   =  0,5\, \unit{\newton}\\
    M_{r1=7}  = 0,495\, \unit{\newton}, \, \, \, \, M_{r2=7}   =  0,7\, \unit{\newton}\\
    M_{r1=9}  = 0,636\, \unit{\newton}, \, \, \, \, M_{r2=9}  ,=  0,9\, \unit{\newton}\\
    M_{r1=11} = 0,779\, \unit{\newton}, \, \, \, \, M_{r2=11}  =  1,1\, \unit{\newton}\\
    M_{r1=13} = 0,919\, \unit{\newton}, \, \, \, \, M_{r2=13}  =  1,3\, \unit{\newton}\\
    M_{r1=15} = 1,060\, \unit{\newton}, \, \, \, \, M_{r2=15}  =  1,5\, \unit{\newton}\\
    M_{r1=17} = 1,202\, \unit{\newton}, \, \, \, \, M_{r2=17}  =  1,7\, \unit{\newton}\\        
    M_{r1=19} = 1,343\, \unit{\newton}, \, \, \, \, M_{r2=19}  =  1,9\, \unit{\newton}\\
    M_{r1=21} = 1,484\, \unit{\newton}, \, \, \, \, M_{r2=21}  =  2,1\, \unit{\newton}\\
    M_{r1=23} = 1,626\, \unit{\newton}, \, \, \, \, M_{r2=23}  =  2,3\, \unit{\newton}\\
    M_{r1=25} = 1,767\, \unit{\newton}. \, \, \, \, M_{r2=25}  =  2,5\, \unit{\newton}\\
\end{align*}





\section{Theorie}

\begin{flushleft}
    Das Trägheitmoment $ I $ ist eine von drei physikalische Komponenten die für das beschreiben einer Rotationsbewegung benötigt ist.
    Das Drehmoment $ M $ und die Winkelbeschleunigung $ \dot{\omega} $ sind die anderen Komponenten. 
    Das Trägheitsmoment lässt sich ausdrücken als:   
\end{flushleft}


\begin{align}
    I = mr^2
    \intertext{für eine punktförmige Masse $ m $ mit dem Abstand $ r $ von der Drehachse.
    Alle Massenelemente eines ausgedehnten Körpers, welche sich um eine Achse drehen, bewegen sich mit gleicher Winkelgeschwindigkeit um die Drehachse.
    Daraus folgt das Gesamtträgheitsmoment}
    I = \sum \limits_{i} r^2_{i} \cdot m_{i} . \notag 
    \intertext{Für infinitisimale Massen gilt} 
    I = \displaystyle\int r^2 \hspace{0.1cm} dm. \notag
    \intertext{Das Drehmoment ist eine lageabhängige Komponente, welche von der Lage der Drehachse abhängt.
    Die Berechnung des Trägheitsmomentes ist einfacher berechenbar für geometrische Körper, wie z.B. eine Kugel oder ein Zylinder, bei denen die Drehachse durch den Schwerpunkt $ S $ verlaufen, als für komplexere Körper (wie z.B. eine Holzpuppe). 
    Hierbei ist es einfacher den Körper in kleinere Segmente aufzuteilen und die berechneten Trägheiten aufeinander zu addieren, wenn sie alle von einer Achse abhängig sind.
    Bei Körpern bei denen die Drehachse nicht durch den Schwerpunkt verläuft, sondern um einen Abstand $a$ parrallel verschoben, ist wird der Satz von Steiner, um das Trägheitsmoment zu berechnen, benutzt}
    I = I_{s} + m \cdot a^2. \label{2}
    \intertext{Das $I_{s}$ ist hierbei das Trägheitsmoment, welches durch den Schwerpunkt des Körpers verläuft,
    $ m $ die Gesamtmasse des Körpers und a der Abstand zur Schwerpunktsachse.
    Einige Trägheitsmomente welche für dieses Experiment wichtig sein werden, sind}
    I_{\text{Stab}} = \frac{1}{12} ml^2, \notag \\ 
    I_{\text{Kugel}} = \frac{2}{5} mR^2, \label{3} \\
    I_{\text{Zylinder}} = \frac{1}{2}mR^2. \label{4}
    \intertext{Für das Drehmoment gilt, wenn eine Kraft im Abstand r eingreift, $ M = \vec{F} \times \vec{r} $.
    Würde diese Kraft dabei um $ \alpha $ ausgelenkt, dann folgt dafür:} 
    \to  M = \vec{F} \cdot \vec{r}  \cdot \sin(\alpha) 
\end{align}

\begin{center}
    ($\vec{F}$ und $\vec{r}$ stehen senkrecht zueinander).
\end{center}

\begin{align}
    \intertext{Durch das Auslenken entsteht eine harmonische Schwingung mit der Dauer:}
    T = 2\pi \cdot \sqrt{\frac{I}{D}}.
    \intertext{Durch Umformen ensteht}
    I = \frac{T^2 \cdot D}{4\pi^2}.
    \intertext{Die Relative Abweichung wird in \% bestimmt, durch}
    a = \frac{x_{exp} - x_{theo}}{x_{theo}} \cdot 100 \label{8}
\end{align}
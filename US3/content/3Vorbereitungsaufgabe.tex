\newpage
\section{Vorbereitungsaufgabe}

\begin{align}
    \intertext{Die Vorbereitungsaufgabe für diesen Versuch besteht aus der Berechnung der Dopplerwinkel $\alpha$ für die Prismenwinkel $\unit{\eta}$ für $15\unit{\degree}$, $30\unit{\degree}$, $60\unit{\degree}.$
    Dafür wird die Gleichung}
    \alpha = 90\unit{\degree} - \arcsin\left(\sin\theta \cdot \frac{c_{\text{L}}}{c_{\text{p}}}\right) \label{5}
    \intertext{verwendet. Dabei sind $c_{\text{L}} = 1800\,\frac{\unit{\meter}}{\unit{\second}}$ und $c_{\text{p}} = 2700\,\frac{\unit{\meter}}{\unit{\second}}$ \cite{a1} }
    \alpha_{15\unit{\degree}} = 90\unit{\degree} - \arcsin\left(\sin 15\unit{\degree} \cdot \frac{1800}{2700}\right) = 80,06\unit{\degree}\notag \\ 
    \notag\\
    \alpha_{30\unit{\degree}} = 90\unit{\degree} - \arcsin\left(\sin 30\unit{\degree} \cdot \frac{1800}{2700}\right) = 70,52\unit{\degree}\notag \\ 
    \notag\\
    \alpha_{60\unit{\degree}} = 90\unit{\degree} - \arcsin\left(\sin 60\unit{\degree} \cdot \frac{1800}{2700}\right) = 54,73\unit{\degree} \notag
\end{align}

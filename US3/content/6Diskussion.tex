\section{Diskussion}

\begin{flushleft}
    Anhand der zwei Versuchsteile konnten verschiedene Erkenntnisse erbracht werden. 
    Beim Vergleich der mit Strömungsgeschwindigkeiten mit den Theoriewerten der einzelnen Flussgeschwindigkeiten, aus dem ersten Teil, folgt
\end{flushleft}

\begin{table}
    \centering
    \caption{Abweichung in \% für die Einstellwinkel}
    \label{Tabelle}
    \begin{tabular} {c | c  c  c}
        \toprule
        {$ \frac{\text{l}}{\text{min}} $} &
        {$ 15\unit{\degree} $} &
        {$ 30\unit{\degree} $} &
        {$ 45\unit{\degree} $} \\
        \midrule
        3.0 & 22,2 & 28,0 & 50,7 \\
        3.5 & 44,5 & 9,40 & 47,2 \\
        4.0 & 14,1 & 42,3 & 43,5 \\
        4.5 & 51,5 & 27,3 & 36,8 \\
        5.0 & 0,90 & 61,3 & 37,7 \\
    \end{tabular}  
\end{table}

\begin{flushleft}
    Auffällig hierbei ist, dass die Abweichungen für einen Einstellwinkel von $15\unit{\degree}$ genug sind unf mit zunehmender Winkel, sich die Abweichung vergrößert.
    Die Zunahme derAbweichung lässt sich auf mögliche Fehlerquellen herleiten, wie durch das Ablesen der Messwerte.
    Es ist zu beachten, dass die Werte beider Messung sich sehr schnell verändert haben, wodurch sich keine genauen Wertepaare ablesen ließen.
    Dadurch entsteht eine Abweichung der Werte und somit eine Ungenauigkeit der Ergebnisse.
    Trotz alledem befinden sich die Werte in einem annehmbaren Bereich, z.B. die geringste Abweichung beträgt $0,9\%$
    und die meisten Werte befinden sich im Abweichungsbereich.
\end{flushleft}

\begin{flushleft}
    Die Fehlerquellen lassen sich ebenso auf den zweiten Versuchsteil herleiten.
    Die Sonde musste präzise gehalten werden da bei einer geringen Bewegung das Messprogramm aufgehört hatte, die Messung durchzuführen.
    Unbeschadet dessen sind die Messergebnisse bzw. Diagramme entsprechend der Erwartungen.
\end{flushleft} 
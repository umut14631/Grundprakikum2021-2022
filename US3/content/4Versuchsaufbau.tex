\section{Versuchsaufbau und Versuchsdurchführung}

\begin{flushleft}
Der Versuchsaufbau zu diesem Versuch besteht aus einem Wasser-Glycerin-Kreislauf, welcher durch eine Pumpe angetrieben wird.
Dieser Kreislauf teilt sich in ca. drei Teile mit verschiedenen Rohrdicken, welche in der Tabelle \ref{Tabelle1} aufgelistet werden, auf.
In diesem Wasser-Glycerin-Gemisch befinden sich Glaskugeln, welche mit der Flüssigkeit durch den Kreislauf strömen.
Auf diesen Strömungsröhren befindet sich jeweils ein Doppler-Prisma mit drei verschiedenen Einschallwinkeln ($15\unit{\degree}$, $30\unit{\degree}$ und $45\unit{\degree}$).
\end{flushleft}

\begin{flushleft}
Dazu wird ebenfalls ein Ultraschall Doppler-Generator mit einer Ultraschallsonde mit einer Frequenz von bis zu $2\,\unit{\mega\hertz}$, ein Rechner zur Datenauswertung und Ultraschallgel benötigt.
Auf dem Rechner sollte das Programm Flowview installiert sein und der Rechner sollte mit dem Generator verbunden sein um die Daten aufnehmen zu können.
Das Ultraschallgel muss auf dem Prisma sowie auf dem Strömungsrohr aufgetragen werden, weil Luft zwischen dem Prisma und dem Strömungsrohr als drittes Medium wirkt und dadurch die Messung manipuliert bzw. gar nicht erst durchgeführt werden kann.
\end{flushleft}
\begin{flushleft}
Eine wichtige Randbemerkung ist zu dem, dass die Ultraschallsonde sehr empfindlich ist und deswegen sehr vorsichtig mit dieser umgehen sollte.
\end{flushleft}

\begin{table}[H]  
    \centering
    \caption{Die Auflistung der verschiedenen Rohrdicken} 
    \label{Tabelle1}
    \begin{tabular} {c  c  c }
        \toprule
        {$  $} &
        {$ \text{Innendurchmesser} $} &
        {$ \text{Außendurchmesser} $} \\
        \midrule
        kleine Rohrdicke   & $7\unit{\milli\meter}$  & $10\unit{\milli\meter}$   \\
        mittlere Rohrdicke & $10\unit{\milli\meter}$ & $15\unit{\milli\meter}$    \\
        große Rohrdicke    & $16\unit{\milli\meter}$ & $20\unit{\milli\meter}$   \\
      
    \end{tabular} 
\end{table}

\begin{flushleft}
    Für den ersten Arbeitsauftrag wird die Strömungsgeschwindigkeit bestimmt, dies geschieht wie folgt.
    Man wählt eines der drei verschieden dicken Rohre und beschmiert dieses mit Ultraschallgel. 
    Auf dieses wird das Prisma gesetzt und auf das Prisma wird wieder Ultraschallgel gegeben.
    Die Ultraschallsonde wird an einem der drei wählbaren Winkeln angesetzt.
    Der Generator wird bei \textit{Sample Volume} auf Large gestellt.
    Flowview wird auf dem Rechner geöffnet und die nötigen Parameter ergänzt, wie z.B. Winkel und Rohrdicke.
    Danach wird gemessen und Werte wie die maximale Frequenz die erzeugt wird, die Flowrate und die Geschwindigkeit der Glaskugeln.
    Die Messung wird für alle drei Winkel durchgeführt mit jeweils 5 verschiedenen Geschwindigkeiten durchgeführt, alle jedoch nur an einem der drei Rohrdicken 
\end{flushleft}

\begin{flushleft}
    Der zweite Arbeitsauftrag besteht aus der Bestimmung des Strömungsprofils des mitteldicken Rohres. 
    Der Aufbau ist der selbe wie davor jedoch werden einige kleine Veränderungen vorgenommen.
    Die Ultraschallsonde wird nur bei dem Prismawinkel von $15\,\unit{\degree}$ befestigt. 
    Der Ultraschall-Generator wird bei \textit{Sample Volume} auf Small gestellt und die Einstellung \textit{Depth} wird in 0,5 Schritten erhöht.
    Begonnen wird bei 12,5 und geht bis 18,5.
    Aufgenommen werden diesmal maximale Frequenz und die Intensität. 
    Dies wird für eine Pumpleistung von $70\%$ und danach mit einer Pumpleistung von $45\%$ durchgeführt.
\end{flushleft}
\section{Diskussion}

\begin{align*}
    \intertext{Im ersten Teil des Versuches wurden die Kennlinien untersucht.
    Die Kennlinie verhalten sich wie in Abbildung \ref{Abbildung3}.
    Es ist anzumerken, dass die messwerte bei der Apparatur höchstens bis $100\,\unit{\volt}$ gingen. 
    Durch eine höhere Abzahl an Messwerten wäre die Asymptote bzw. der Sättigungsstrom besser bestimmbar.
    Der Sättigungsstrom an der Kennlinie sechs ist nicht direkt erkennbar, deswegen wurde der Wendepunkt abgeschätzt.
    Zudem stellt das Ablesen ebenso eine Ungenauigkeit das, denn bei $50\,\unit{\volt}$ sprang der Zeiger am Messgerät von dem einen auf den anderen.}
    \intertext{Auffällig bei der Bestimmung des Exponenten b, welche zur Überprüfung des Raumladungsgesetzes dient, weist die Abweichung auf:}
    \text{b}_{\text{exp}} = 1,37 \pm 0,03\,\,\,\,\,\,\,\text{b}_{\text{theo}} = 1,5\\
    \to 8,6\% 
    \intertext{Die Abweichung könnte auf den Ablesefehler oder auf nicht genügend Messwerte deuten.}
    \intertext{Die Glühkathode kann durch einen Strom von $1000$ bis $3000\,\unit{\kelvin}$ erhitzt werden.
    Der berechnete Wert für die Temperatur entspricht $\text{T}_{\text{exp}} = (2000,51\pm50,86)\,\unit{\kelvin}$ und liegt somit im Intervall.
    Die Kurve verhält sich, wie erwartet, exponentiell.
    Die Kathodentemperatur aus den Kennlinien liegen ebenso im Intervall. 
    Es liegt eine positive Gegenspannung an, was dafür sorgt, dass der gemessene Strom fällt. }
    \intertext{Die berechnete Austrittsarbeit von Wolfram weist die Abweichung auf:}
    \overline{\text{W}}_{\text{A}} = ( 5,87 \pm 0,18 )\,\text{eV} \,\,\,\,\,\,\,\, \text{W}_{\text{A, theo}} = 4,55\,\text{eV}\,\, \text{\cite{a2}} \\
    \to 29\%
    \intertext{Die Abweichung führt auf die abgeschätzten Sättigungsstrom.
    Nichtsdestotrotz eignet sich der Versuch für die Bestimmung der Austrittsarbeit von Wolfram.
    Jedoch sollte die Messapparatur funktionstüchtig sein und für mehrere Messungen eignen.}
\end{align*}